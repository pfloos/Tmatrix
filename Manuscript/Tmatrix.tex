\documentclass[aip,jcp,reprint,noshowkeys,superscriptaddress]{revtex4-1}
\usepackage{graphicx,dcolumn,bm,xcolor,microtype,multirow,amscd,amsmath,amssymb,amsfonts,physics,longtable,wrapfig,txfonts}
\usepackage[version=4]{mhchem}

\usepackage[utf8]{inputenc}
\usepackage[T1]{fontenc}
\usepackage{txfonts}

\usepackage[
	colorlinks=true,
    citecolor=blue,
    breaklinks=true
	]{hyperref}
\urlstyle{same}

\newcommand{\ie}{\textit{i.e.}}
\newcommand{\eg}{\textit{e.g.}}
\newcommand{\alert}[1]{\textcolor{red}{#1}}
\usepackage[normalem]{ulem}
\newcommand{\titou}[1]{\textcolor{red}{#1}}
\newcommand{\trashPFL}[1]{\textcolor{red}{\sout{#1}}}
\newcommand{\trashXB}[1]{\textcolor{darkgreen}{\sout{#1}}}
\newcommand{\PFL}[1]{\titou{(\underline{\bf PFL}: #1)}}

\newcommand{\mc}{\multicolumn}
\newcommand{\fnm}{\footnotemark}
\newcommand{\fnt}{\footnotetext}
\newcommand{\tabc}[1]{\multicolumn{1}{c}{#1}}
\newcommand{\SI}{\textcolor{blue}{supporting information}}
\newcommand{\QP}{\textsc{quantum package}}
\newcommand{\T}[1]{#1^{\intercal}}

% coordinates
\newcommand{\br}{\mathbf{r}}
\newcommand{\bx}{\mathbf{x}}

% methods
\newcommand{\evGW}{ev$GW$}	
\newcommand{\qsGW}{qs$GW$}	
\newcommand{\GOWO}{$G_0W_0$}	
\newcommand{\Hxc}{\text{Hxc}}
\newcommand{\xc}{\text{xc}}
\newcommand{\Ha}{\text{H}}
\newcommand{\co}{\text{c}}
\newcommand{\x}{\text{x}}

% 
\newcommand{\Norb}{N_\text{orb}}
\newcommand{\Nocc}{O}
\newcommand{\Nvir}{V}

% operators
\newcommand{\hH}{\Hat{H}}
\newcommand{\hS}{\Hat{S}}

% methods
\newcommand{\KS}{\text{KS}}
\newcommand{\HF}{\text{HF}}
\newcommand{\RPA}{\text{RPA}}
\newcommand{\ppRPA}{\text{pp-RPA}}
\newcommand{\BSE}{\text{BSE}}
\newcommand{\dBSE}{\text{dBSE}}
\newcommand{\GW}{GW}
\newcommand{\GT}{GT}
\newcommand{\stat}{\text{stat}}
\newcommand{\dyn}{\text{dyn}}
\newcommand{\TDA}{\text{TDA}}

% energies
\newcommand{\Enuc}{E^\text{nuc}}
\newcommand{\Ec}{E_\text{c}}
\newcommand{\EHF}{E^\text{HF}}
\newcommand{\EBSE}{E^\text{BSE}}
\newcommand{\EcRPA}{E_\text{c}^\text{RPA}}
\newcommand{\EcBSE}{E_\text{c}^\text{BSE}}

% orbital energies
\newcommand{\e}[2]{\eps_{#1}^{#2}}
\newcommand{\eHF}[1]{\eps^\text{HF}_{#1}}
\newcommand{\eKS}[1]{\eps^\text{KS}_{#1}}
\newcommand{\eQP}[1]{\eps^\text{QP}_{#1}}
\newcommand{\eGOWO}[1]{\eps^\text{\GOWO}_{#1}}
\newcommand{\eGW}[1]{\eps^{GW}_{#1}}
\newcommand{\eevGW}[1]{\eps^\text{\evGW}_{#1}}
\newcommand{\eGnWn}[2]{\eps^\text{\GnWn{#2}}_{#1}}
\newcommand{\Om}[2]{\Omega_{#1}^{#2}}
\newcommand{\tOm}[2]{\Tilde{\Omega}_{#1}^{#2}}



% Matrix elements
\newcommand{\tA}[2]{\Tilde{A}_{#1}^{#2}}
\renewcommand{\S}[1]{S_{#1}}
\newcommand{\ABSE}[2]{A_{#1}^{#2,\text{BSE}}}
\newcommand{\BBSE}[2]{B_{#1}^{#2,\text{BSE}}}
\newcommand{\ARPA}[2]{A_{#1}^{#2,\text{RPA}}}
\newcommand{\BRPA}[2]{B_{#1}^{#2,\text{RPA}}}
\newcommand{\ARPAx}[2]{A_{#1}^{#2,\text{RPAx}}}
\newcommand{\BRPAx}[2]{B_{#1}^{#2,\text{RPAx}}}
\newcommand{\G}[1]{G_{#1}}
\newcommand{\LBSE}[1]{L_{#1}}
\newcommand{\XiBSE}[1]{\Xi_{#1}}
\newcommand{\Po}[1]{P_{#1}}
\newcommand{\W}[2]{W_{#1}^{#2}}
\newcommand{\tW}[2]{\widetilde{W}_{#1}^{#2}}
\newcommand{\Wc}[1]{W^\text{c}_{#1}}
\newcommand{\vc}[1]{v_{#1}}
\newcommand{\Sig}[2]{\Sigma_{#1}^{#2}}
\newcommand{\SigC}[1]{\Sigma^\text{c}_{#1}}
\newcommand{\SigX}[1]{\Sigma^\text{x}_{#1}}
\newcommand{\SigXC}[1]{\Sigma^\text{xc}_{#1}}
\newcommand{\Z}[1]{Z_{#1}}
\newcommand{\MO}[1]{\phi_{#1}}
\newcommand{\SO}[1]{\psi_{#1}}
\newcommand{\ERI}[2]{(#1|#2)}
\newcommand{\rbra}[1]{(#1|}
\newcommand{\rket}[1]{|#1)}
\newcommand{\sERI}[2]{[#1|#2]}

%% bold in Table
\newcommand{\bb}[1]{\textbf{#1}}
\newcommand{\rb}[1]{\textbf{\textcolor{red}{#1}}}
\newcommand{\gb}[1]{\textbf{\textcolor{darkgreen}{#1}}}

% excitation energies
\newcommand{\OmRPA}[1]{\Omega_{#1}^{\text{RPA}}}
\newcommand{\OmRPAx}[1]{\Omega_{#1}^{\text{RPAx}}}
\newcommand{\OmBSE}[1]{\Omega_{#1}^{\text{BSE}}}


% Matrices
\newcommand{\bO}{\mathbf{0}}
\newcommand{\bI}{\mathbf{1}}
\newcommand{\bvc}{\mathbf{v}}
\newcommand{\bSig}{\mathbf{\Sigma}}
\newcommand{\bSigX}{\mathbf{\Sigma}^\text{x}}
\newcommand{\bSigC}{\mathbf{\Sigma}^\text{c}}
\newcommand{\bSigGW}{\mathbf{\Sigma}^{GW}}
\newcommand{\be}{\mathbf{\epsilon}}
\newcommand{\beGW}{\mathbf{\epsilon}^{GW}}
\newcommand{\beGnWn}[1]{\mathbf{\epsilon}^\text{\GnWn{#1}}}
\newcommand{\bde}{\mathbf{\Delta\epsilon}}
\newcommand{\bdeHF}{\mathbf{\Delta\epsilon}^\text{HF}}
\newcommand{\bdeGW}{\mathbf{\Delta\epsilon}^{GW}}
\newcommand{\bOm}[1]{\mathbf{\Omega}^{#1}}
\newcommand{\bA}[2]{\mathbf{A}_{#1}^{#2}}
\newcommand{\bB}[2]{\mathbf{B}_{#1}^{#2}}
\newcommand{\bC}[2]{\mathbf{C}_{#1}^{#2}}
\newcommand{\bX}[2]{\mathbf{X}_{#1}^{#2}}
\newcommand{\bY}[2]{\mathbf{Y}_{#1}^{#2}}
\newcommand{\bZ}[2]{\mathbf{Z}_{#1}^{#2}}
\newcommand{\bK}{\mathbf{K}}
\newcommand{\bP}[1]{\mathbf{P}^{#1}}

% units
\newcommand{\IneV}[1]{#1 eV}
\newcommand{\InAU}[1]{#1 a.u.}
\newcommand{\InAA}[1]{#1 \AA}
\newcommand{\kcal}{kcal/mol}

% orbitals, gaps, etc
\newcommand{\eps}{\varepsilon}
\newcommand{\IP}{I}
\newcommand{\EA}{A}
\newcommand{\HOMO}{\text{HOMO}}
\newcommand{\LUMO}{\text{LUMO}}
\newcommand{\Eg}{E_\text{g}}
\newcommand{\EgFun}{\Eg^\text{fund}}
\newcommand{\EgOpt}{\Eg^\text{opt}}
\newcommand{\EB}{E_B}

\newcommand{\sig}{\sigma}
\newcommand{\bsig}{{\Bar{\sigma}}}
\newcommand{\sigp}{{\sigma'}}
\newcommand{\bsigp}{{\Bar{\sigma}'}}
\newcommand{\taup}{{\tau'}}

\newcommand{\up}{\uparrow}
\newcommand{\dw}{\downarrow}
\newcommand{\upup}{\uparrow\uparrow}
\newcommand{\updw}{\uparrow\downarrow}
\newcommand{\dwup}{\downarrow\uparrow}
\newcommand{\dwdw}{\downarrow\downarrow}
\newcommand{\spc}{\text{sc}}
\newcommand{\spf}{\text{sf}}
\newcommand{\ssp}{\text{ss}}
\newcommand{\osp}{\text{os}}

% addresses
\newcommand{\LCPQ}{Laboratoire de Chimie et Physique Quantiques (UMR 5626), Universit\'e de Toulouse, CNRS, UPS, France}
\newcommand{\LPT}{Laboratoire de Physique Th\'eorique, Universit\'e de Toulouse, CNRS, UPS, France}
\newcommand{\ETSF}{European Theoretical Spectroscopy Facility (ETSF)}
\begin{document}	

\title{Static and Dynamic Bethe-Salpeter Equations in the $T$-Matrix Approximation}
\author{Pierre-Fran\c{c}ois \surname{Loos}}
	\email{loos@irsamc.ups-tlse.fr}
	\affiliation{\LCPQ}
\author{Pina Romaniello}
	\email{romaniello@irsamc.ups-tlse.fr}
	\affiliation{\LPT}
	\affiliation{\ETSF}

\begin{abstract}
While the well-established $GW$ approximation corresponds to a resummation of the direct ring diagrams and is particularly well suited for weakly-correlated systems, the $T$-matrix approximation does sum ladder diagrams up to infinity and is supposedly more appropriate in the presence of strong correlation.
Here, we derive and implement, for the first time, the static and dynamic Bethe-Salpeter equations when one considers $T$-matrix quasiparticle energies as well as a $T$-matrix-based kernel.
The performance of the static scheme and its perturbative dynamical correction are assessed by computed the neutral excited states of molecular systems.
Comparison with more conventional schemes as well as other wave function methods are also reported.
%\bigskip
%\begin{center}
%	\boxed{\includegraphics[width=0.5\linewidth]{TOC}}
%\end{center}
%\bigskip
\end{abstract}

\maketitle

The $GW$ approximation \cite{Hedin_1965} of many-body perturbation theory \cite{Martin_2016} is becoming a method of choice to target charged excitations (\ie, ionization potentials and electron affinities) in molecular systems. \cite{Aryasetiawan_1998,Onida_2002,Reining_2017,Golze_2019,Bruneval_2021} 
These so-called quasiparticle energies can be experimentally measured from direct and inverse photoemission spectroscopies.
From a more theoretical point of view, $GW$ corresponds to an elegant resummation of all direct ring diagrams from the particle-hole (ph) channel which is particularly justified in the high-density or weakly-correlated regime. \cite{Gell-Mann_1957,Nozieres_1958}
Within the $GW$ approximation, the self-energy --- one of the key quantities of Hedin's equations \cite{Hedin_1965} --- reads
\begin{equation}
	\Sigma^{GW}(1,2) = i G(1,2) W(1,2)
\end{equation}
where $G$ is the one-body Green's function, $W$ is the dynamically-screened Coulomb potential, and, \eg, $1 \equiv (\sigma_1,\br_1,t_1)$ is a composite coordinate gathering spin, space, and time variables.

Alternatives to $GW$ do exist. For example, the $T$-matrix (or Bethe-Goldstone) approximation, first introduced in nuclear physics, \cite{Bethe_1957,Baym_1961,Baym_1962,Danielewicz_1984a,Danielewicz_1984b} then in condensed matter physics, \cite{Liebsch_1981,Bickers_1989,Bickers_1991,Katsnelson_1999,Katsnelson_2002,Zhukov_2005,vonFriesen_2010,Romaniello_2012,Gukelberger_2015} and more recently in quantum chemistry, \cite{Zhang_2017,Li_2021b} sums to infinity the ladder diagrams from the particle-particle (pp) channel and is justified in the low-density or strongly-correlated regime. \cite{Danielewicz_1984a,Danielewicz_1984b,Liebsch_1981,Shepherd_2014}
While the two-point screened interaction $W$ is the cornerstone of $GW$, the $T$-matrix approximation relies on a more complex (four-point) effective interaction --- the so-called $T$ matrix --- yielding the following self-energy: 
\begin{equation}
	\Sigma^{GT}(1,2) = i \int G(4,3) T(1,3;2,4) d3 d4
\end{equation}
The natural idea of combining the ph and pp channels is also possible and has been explored, for example, in the Hubbard dimer within many-body perturbation theory \cite{Romaniello_2012} and the uniform electron gas \cite{Loos_2016} within coupled-cluster theory. \cite{Shepherd_2014}

One of the key features of the $T$-matrix approximation is its exactness up to the second order thanks to the inclusion of second-order exchange diagrams.
This class of diagrams, which are particularly important in molecular systems and explain the improvement brought by the second-order screened exchange (SOSEX) correction applied to $GW$, \cite{Romaniello_2009a,Ren_2015,Loos_2018b} are well known to be missing in the $GW$ approximation.

Let us consider closed-shell electronic systems consisting of $N$ electrons and $K$ one-electron basis functions.
The number of singly-occupied and virtual (\ie, unoccupied) spinorbitals are $O = N$ and $V = K - O$, respectively.
Let us denote as $\SO{p}(\bx)$ the $p$th spinorbital and $\e{p}{}$ its one-electron energy.
The composite variable $\bx = (\sigma,\br)$ gathers spin ($\sigma$) and spatial ($\br$) variables.
We assume real quantities throughout this manuscript, $i$, $j$, $k$, and $l$ are occupied orbitals, $a$, $b$, $c$, and $d$ are unoccupied orbitals, $p$, $q$, $r$, and $s$ indicate arbitrary orbitals, $m$ labels single excitations, while $n$ labels double electron attachments or double electron detachments.

By definition, the one-body Green's function is \cite{Martin_2016}
\begin{equation}
\label{eq:G}
	G(\bx_1,\bx_2;\omega) 
	= \sum_i \frac{\SO{i}(\bx_1) \SO{i}(\bx_2)}{\omega - \e{i}{} - i\eta}	
	+ \sum_a \frac{\SO{a}(\bx_1) \SO{a}(\bx_2)}{\omega - \e{a}{} + i\eta}	
\end{equation}
where $\eta$ is a positive infinitesimal, and its nature is completely defined by the set of orbitals and corresponding energies that are used to build it.
For example, $G^{\HF}(\bx_1,\bx_2;\omega)$ is the Hartree-Fock (HF) Green's function built from the HF spinorbitals $\SO{p}^{\HF}(\bx)$ and energies $\e{p}{\HF}$.

Contrary to the $GW$ approximation which relies on the (two-point) dynamically-screened Coulomb potential $W$ computed from a ph-random-phase approximation (ph-RPA) problem to target charged excitations, \cite{Hedin_1965,Aryasetiawan_1998,Onida_2002,Martin_2016,Reining_2017,Golze_2019} here we consider the $GT$ approximation where one employs the (four-point) $T$ matrix obtained from solving the pp-RPA equations.

The non-Hermitian pp-RPA problem reads \cite{Schuck_Book,vanAggelen_2013,Peng_2013,Scuseria_2013,Yang_2013,Yang_2013b,vanAggelen_2014,Yang_2014a,Zhang_2015,Zhang_2016}
\begin{equation}
\label{eq:LR-RPA}
	\begin{pmatrix}
		\bA{}{\ppRPA}			&	\bB{}{\ppRPA}	\\
		-\T{(\bB{}{\ppRPA})}	&	-\bC{}{\ppRPA}	\\
	\end{pmatrix}
	\cdot
	\begin{pmatrix}
		\bX{n}{N\pm2}	\\
		\bY{n}{N\pm2}	\\
	\end{pmatrix}
	=
	\Om{n}{N\pm2}
	\begin{pmatrix}
		\bX{n}{N\pm2}	\\
		\bY{n}{N\pm2}	\\
	\end{pmatrix}
\end{equation}
where the elements of the various matrices are defined as
\begin{subequations}
\begin{align}
	A_{ab,cd}^{\ppRPA} & = \delta_{ab} \delta_{cd} (\e{a}{} + \e{b}{}) + \mel{ab}{}{cd}
	\\ 
	B_{ab,ij}^{\ppRPA} & = \mel{ab}{}{ij}
	\\ 
	C_{ij,kl}^{\ppRPA} & = - \delta_{ik} \delta_{jl} (\e{i}{} + \e{j}{}) +\mel{ij}{}{kl}
\end{align}
\end{subequations}
and 
\begin{equation}
	\mel{pq}{}{rs} = \braket{pq}{rs} - \braket{pq}{sr}
\end{equation}
are two-electron integrals in the spinorbital basis, \ie,
\begin{equation}
	\braket{pq}{rs} = \iint \SO{p}(\bx_1) \SO{q}(\bx_1) \frac{1}{\abs{\br_1 - \br_2}} \SO{r}(\bx_2) \SO{s}(\bx_2)  d\bx_1 d\bx_2
\end{equation}
The pp-RPA problem yields, in the absence of instabilities (which should not appear in Coulombic systems with repulsive interactions only \cite{Scuseria_2013}), $V(V-1)/2$ positive eigenvalues $\Om{n}{N+2}$ and $O(O-1)/2$ negative eigenvalues $\Om{n}{N-2}$, which  correspond respectively to double attachments and double detachments.
The pp-RPA correlation energies is given by \cite{Peng_2013,Scuseria_2013}
\begin{equation}
\begin{split}
	\Ec^\ppRPA 
	& = + \sum_n \Om{n}{N+2} - \Tr(\bA{}{\ppRPA}) 
	\\
	& = - \sum_n \Om{n}{N-2} - \Tr(\bC{}{\ppRPA})
\end{split}
\end{equation}

Based on the knowledge of the pp-RPA eigenvalues and eigenvectors, one can then construct the $T$ matrix.
In the spinorbital basis, the correlation part of the $T$ matrix has the following expression \cite{Zhang_2016}
\begin{equation}
\label{eq:T}
	T_{pq,rs}(\omega) 
		= \sum_{n} \frac{\braket*{pq}{\chi_n^{N+2}}\braket*{rs}{\chi_n^{N+2}}}{\omega - \Om{n}{N+2} + i\eta}
		- \sum_{n} \frac{\braket*{pq}{\chi_n^{N-2}}\braket*{rs}{\chi_n^{N-2}}}{\omega - \Om{n}{N-2} - i\eta}
\end{equation}
with
\begin{subequations}
\begin{align}
	\braket*{pq}{\chi_n^{N+2}} & = \sum_{c < d} \mel{pq}{}{cd} X_{cd,n}^{N+2} + \sum_{k < l}  \mel{pq}{}{kl} Y_{kl,n}^{N+2}
	\\
	\braket*{pq}{\chi_n^{N-2}} & = \sum_{c < d} \mel{pq}{}{cd} X_{cd,n}^{N-2} + \sum_{k < l}  \mel{pq}{}{kl} X_{kl,n}^{N-2}
\end{align}
\end{subequations}

Combining Eqs.~\eqref{eq:G} and \eqref{eq:T}, the correlation part of the $T$-matrix self-energy reads \cite{Romaniello_2012,Martin_2016,Zhang_2017,Li_2021b}
\begin{equation}
\label{eq:SigGT}
	\Sigma^{GT}_{pq}(\omega)
	= \sum_{in} \frac{\braket*{pi}{\chi_n^{N+2}}\braket*{qi}{\chi_n^{N+2}}}{\omega + \e{i}{} - \Om{n}{N+2} + i\eta}
	- \sum_{an} \frac{\braket*{pa}{\chi_n^{N-2}}\braket*{qa}{\chi_n^{N-2}}}{\omega + \e{a}{} - \Om{n}{N-2} - i\eta}
\end{equation}
While the dynamical $GW$ self-energy corresponds to the downfolding of the 2h1p and 2p1h configurations on the 1h and 1p configurations via their coupling with the 1h1p configurations, respectively, \cite{Bintrim_2021a} Eq.~\eqref{eq:SigGT} shows that, in the case of the $T$-matrix approximation, the same 2h1p and 2p1h configurations are downfolded on the 1p and 1h configurations via their coupling with the 2h and 2p configurations, respectively.

Within the (perturbative) one-shot $GT$ scheme (labeled as $G_0T_0$ in the following), the quasiparticle energies are obtained via linearization of the quasiparticle equation, \cite{Strinati_1980,Hybertsen_1985a,Hybertsen_1986,Godby_1988,Linden_1988,Northrup_1991,Blase_1994,Rohlfing_1995,Shishkin_2007} \ie,
\begin{equation}
\label{eq:G0T0}
	\e{p}{G_0T_0} = \e{p}{\HF} + Z_p \Sigma^{GT}_{pp}(\e{p}{\HF})
\end{equation}
where we have assumed a HF starting point and
\begin{equation}
	Z_p = \qty[ 1 - \eval{\pdv{\Sigma^{T}_{pp}(\omega)}{\omega}}_{\omega = \e{p}{\HF}} ]^{-1}
\end{equation}
is the renormalization factor.
Other levels of self-consistency can be considered like the partially self-consistent \textit{``eigenvalue''} $GT$ (ev$GT$)  \cite{Hybertsen_1986,Shishkin_2007,Blase_2011,Faber_2011,Rangel_2016,Gui_2018} or the quasiparticle self-consistent $GT$ (qs$GT$) \cite{Faleev_2004,vanSchilfgaarde_2006,Kotani_2007,Ke_2011,Kaplan_2016} schemes.

Like the one-body Green's function is the pillar of the $GW$ and $GT$ approximations, the two-body Green's function $G_2$ is the key quantity of the Bethe-Salpeter equation (BSE) formalism of many-body perturbation theory \cite{Salpeter_1951,Strinati_1988,Blase_2018,Blase_2020} via its link with the two-body correlation function $L$ which satisfies the self-consistent BSE 
\begin{multline}
	 iL(1,2;1',2') = L_0(1,2;1',2') 
	 \\
	 + \int L_0(1,4;1',3) \Xi(3,5;4,6) L(6,2;5,2') d3d4d5d6
\end{multline}
where
\begin{subequations}
\begin{align}
	 iL_0(1,4;1',3) & = G(1,3) G(4,1')
	 \\
	 iL(1,2;1',2') & = -G_2(1,2;1',2') + G(1,1')G(2,2')
\end{align}
\end{subequations}
and 
\begin{equation}
\label{eq:Xi}
	\Xi(3,5;4,6) = i \fdv{\Sigma(3,4)}{G(6,5)}
\end{equation}
is the so-called BSE kernel that takes into account the self-consistent variation of $\Sigma$ with respect to the variation of $G$.
By taking into account the interaction of the excited electron and its hole left behind (the infamous excitonic effect), the BSE is able to faithfully model (neutral) optical excitations as measured by absorption spectroscopy. 
The moderate cost of the BSE [which scales as $\order*{K^4}$ in its standard implementation] and its all-round accuracy are the main reasons behind its growing popularity in the molecular electronic structure community. \cite{Rohlfing_1999a,Horst_1999,Puschnig_2002,Tiago_2003,Boulanger_2014,Jacquemin_2015a,Bruneval_2015,Jacquemin_2015b,Hirose_2015,Jacquemin_2017a,Jacquemin_2017b,Rangel_2017,Krause_2017,Gui_2018,Blase_2018,Liu_2020,Blase_2020,Holzer_2018a,Holzer_2018b,Loos_2020e,Loos_2021}

In order to target neutral (singly-)excited states, we first consider the static version of the BSE employing the $GT$ quasiparticle energies [see Eq.~\eqref{eq:G0T0}] as well as $T$-matrix kernel [\ie, $\Sigma^{\GT}$ in Eq.~\eqref{eq:Xi}].
In this case, the BSE@$GT$ linear eigenvalue problem simply reads
\begin{equation}
\label{eq:BSE}
	\begin{pmatrix}
		\bA{}{\BSE}		&	\bB{}{\BSE}	\\
		-\bB{}{\BSE}	&	-\bA{}{\BSE}	\\
	\end{pmatrix}
	\cdot
	\begin{pmatrix}
		\bX{m}{\BSE}	\\
		\bY{m}{\BSE}	\\
	\end{pmatrix}
	=
	\Om{m}{\BSE}
	\begin{pmatrix}
		\bX{m}{\BSE}	\\
		\bY{m}{\BSE}	\\
	\end{pmatrix}
\end{equation}
with 
\begin{subequations}
\begin{align}
	A_{ia,jb}^{\BSE} & = \delta_{ij} \delta_{ab} (\e{a}{\GT} - \e{i}{\GT}) + \mel{ib}{}{aj} - T_{ib,aj}(\omega=0)
	\\ 
	B_{ia,jb}^{\BSE} & = \mel{ij}{}{ab} - T_{ij,ab}(\omega=0)
\end{align}
\end{subequations}
The eigenvalues $\Om{m}{\BSE}$ of Eq.~\eqref{eq:BSE} provide $OV$ singlet (\ie, spin-conserved) and $OV$ triplet  (\ie, spin-flip) single excitations.
Note that the spin structure of the BSE@$GT$ equations is analogous to the BSE@$GW$ version, \cite{Monino_2021} and one can compute separately singlet and triplet excitation energies.

Due to the frequency-independent nature of the static BSE, it is well known that one cannot access double (and higher) excitations.
\cite{Loos_2019,Romaniello_2009b,Sangalli_2011,Loos_2020h,Authier_2020,Monino_2021}
In order to go beyond the static approximation, it is possible to consider,  within the dynamical Tamm-Dancoff approximation (which neglects the frequency dependence of the coupling block $\bB{}{}$), the dynamical version of the BSE (dBSE). \cite{Strinati_1988,Romaniello_2009b,Loos_2020h}
In this case, one must solve the (non-linear) dynamical eigenvalue problem
\begin{equation}
\label{eq:dBSE}
	\begin{pmatrix}
		\bA{}{\dBSE}(\Om{S}{})	&	\bB{}{\BSE}	\\
		-\bB{}{\BSE}	&	-\bA{}{\dBSE}(-\Om{S}{})	\\
	\end{pmatrix}
	\cdot
	\begin{pmatrix}
		\bX{S}{\dBSE}	\\
		\bY{S}{\dBSE}	\\
	\end{pmatrix}
	=
	\Om{S}{}
	\begin{pmatrix}
		\bX{S}{\dBSE}	\\
		\bY{S}{\dBSE}	\\
	\end{pmatrix}
\end{equation}
with 
\begin{equation}
	A_{ia,jb}^{\dBSE}(\omega) = \delta_{ij} \delta_{ab} (\e{a}{\GT} - \e{i}{\GT}) + \mel{ib}{}{aj} - \widetilde{T}_{ib,aj}(\omega)
\end{equation}
where, following Strinati's seminal work, \cite{Strinati_1988} one can derive the following expression for the elements of the dynamical $T$ matrix
\begin{equation}
\label{eq:dT}
\begin{split}
	\widetilde{T}_{pq,rs}(\omega) 
		& = \sum_{n} \frac{\braket*{pq}{\chi_n^{N+2}}\braket*{rs}{\chi_n^{N+2}}}{\omega - \Om{n}{N+2} + (\e{i}{\GT} + \e{j}{\GT}) + i\eta}
		\\
		& + \sum_{n} \frac{\braket*{pq}{\chi_n^{N-2}}\braket*{rs}{\chi_n^{N-2}}}{\omega + \Om{n}{N-2} - (\e{a}{\GT} + \e{b}{\GT}) + i\eta}
\end{split}
\end{equation}
from which, one can check that we recover the static expression \eqref{eq:T} in the limit $\Om{n}{N\pm2} \to \infty$.
Equation \eqref{eq:dT} highlights the interesting dynamical structure of the $T$ matrix, where, similarly to the dBSE@$GW$ scheme, \cite{Strinati_1988,Romaniello_2009b,Loos_2020h} the 2h2p configurations are downfolded on the 1h1p configurations. \cite{Bintrim_2021b}

Because solving a non-linear eigenvalue problem is computationally challenging, here we rely on the perturbative scheme developed in Ref.~\onlinecite{Loos_2020h} in order to access \textit{dynamically-corrected} single excitations for which additional relaxation effects coming from higher excitations are taken into account.
\cite{Rohlfing_2000,Romaniello_2009b,Ma_2009a,Ma_2009b,Zhang_2013,Rebolini_2016,Olevano_2019,Loos_2020h,Authier_2020,Monino_2021}
Below, we quickly recap this dynamical perturbative scheme.

Based on Rayleigh-Schr\"odinger perturbation theory, the non-linear eigenproblem \eqref{eq:dBSE} can be split as a zeroth-order static reference and a first-order dynamic perturbation, such that
\begin{multline}
\label{eq:LR-PT}
	\begin{pmatrix}
		\bA{}{\dBSE}(\Om{S}{})		&	\bB{}{\BSE}(\Om{S}{})	\\
		-\bB{}{\BSE}(-\Om{S}{})	&	-\bA{}{\BSE}(-\Om{S}{})	\\
	\end{pmatrix}
	\\
	=
	\begin{pmatrix}
		\bA{}{\BSE}	&	\bB{}{\BSE}	
		\\
		-\bB{}{\BSE}	&	-\bA{}{\BSE}	
		\\
	\end{pmatrix}
	+
	\begin{pmatrix}
		\bA{}{(1)}(\Om{S}{})		&	\bO	\\
		\bO	&	-\bA{}{(1)}(-\Om{S}{})	\\
	\end{pmatrix}
\end{multline}
with
\begin{equation}
	\label{eq:BSE-A1}
	A_{ia,jb}^{(1)}(\omega) = \widetilde{T}_{ib,aj}(\omega) - T_{ib,aj}(\omega = 0)
\end{equation}
As usual, one can naturally expand the $S$th BSE excitation energy and its corresponding eigenvector as
\begin{subequations}
\begin{gather}
	\Om{S}{} = \Om{S}{\BSE} + \Om{S}{(1)} + \ldots,
	\\
	\begin{pmatrix}
		\bX{S}{}	\\
		\bY{S}{}	\\
	\end{pmatrix}
	= 
	\begin{pmatrix}
		\bX{S}{\BSE}	\\
		\bY{S}{\BSE}	\\
	\end{pmatrix}
	+
	\begin{pmatrix}
		\bX{S}{(1)}	\\
		\bY{S}{(1)}	\\
	\end{pmatrix}
	+ \ldots
\end{gather}
\end{subequations}
Solving the static BSE [see Eq.~\eqref{eq:BSE}] yields the (zeroth-order) static $\Om{S}{\BSE}$ excitation energies and their corresponding eigenvectors $\bX{S}{\BSE}$ and $\bY{S}{\BSE}$. 
The first-order correction to the $S$th excitation energy is, within the dynamical Tamm-Dancoff approximation,
\begin{equation}
\label{eq:Om1-TDA}
	\Om{S}{(1)} = \T{(\bX{S}{\BSE})} \cdot \bA{}{(1)}(\Om{S}{\BSE}) \cdot \bX{S}{\BSE}
\end{equation}
This correction can be renormalized by computing, at no extra cost, the renormalization factor which reads
\begin{equation}
\label{eq:Z}
	\zeta_{S} = \qty[ 1 - \T{(\bX{S}{\BSE})} \cdot \left. \pdv{\bA{}{(1)}(\Om{S}{})}{\Om{S}{}} \right|_{\Om{S}{} = \Om{S}{\BSE}} \cdot \bX{S}{\BSE} ]^{-1}
\end{equation}
This yields our final expression for the dynamically-corrected BSE excitation energies:
\begin{equation}
	\Om{S}{\text{dyn}} = \Om{S}{\text{stat}} + \Delta\Om{S}{\text{dyn}} = \Om{S}{\BSE} + \zeta_{S} \Om{S}{(1)}
\end{equation}
Note again that the present perturbative scheme does not allow to access double excitations as only excitations calculated within the static approach can be dynamically corrected.

The present formalism has been implemented in the software package \texttt{QuAcK} \cite{QuAcK} that is freely available on \texttt{github}. 
We consider here only systems with closed-shell singlet ground states.
Thus, the $GW$ and $GT$ calculations are performed by considering a (restricted) HF starting point and standard gaussian basis sets (defined with cartesian functions) are employed.
Note that all quasiparticle energies which are obtained via Eq.~\eqref{eq:G0T0} are corrected in the same way.
Finally, the infinitesimal $\eta$ is set to zero for all calculations.
The ev$GT$ and qs$GT$ schemes have been also implemented but are not considered here.
For the sake of completeness, implementable expressions in the spatial orbital basis are reported in the supporting information.


In terms of computational, the overall scaling of BSE@$GT$ is equivalent to BSE@$GW$ as they both correspond to seeking the lowest eigenvalues of a matrix of size $(\Nocc \Nvir \times \Nocc \Nvir)$. 
However, the prefactor corresponding to the calculations of the $T$-matrix and more specifically to the eigenvalue of the \titou{T2: Discuss the computational cost of the BSE@$GT$ equations, especially compare to BSE@$GW$.}
%In terms of computational cost, if one decides to compute the dynamical correction of the $M$ lowest excitation energies, one must perform, first, a conventional (static) BSE calculation and extract the $M$ lowest eigenvalues and their corresponding eigenvectors [see Eq.~\eqref{eq:LR-BSE-stat}].
%These are then used to compute the first-order correction from Eq.~\eqref{eq:Om1-TDA}, which also require to construct and evaluate the dynamical part of the BSE Hamiltonian for each excitation one wants to dynamically correct. 
%The static BSE Hamiltonian is computed once during the static BSE calculation and does not dependent on the targeted excitation.
%
%Searching iteratively for the lowest eigenstates, via Davidson's algorithm for instance, can be performed in $\order*{\Norb^4}$ computational cost.
%Constructing the static and dynamic BSE Hamiltonians is much more expensive as it requires the complete diagonalization of the $(\Nocc \Nvir \times \Nocc \Nvir)$ RPA linear response matrix [see Eq.~\eqref{eq:LR-RPA}], which corresponds to a $\order*{\Nocc^3 \Nvir^3} = \order*{\Norb^6}$ computational cost. 
%Although it might be reduced to $\order*{\Norb^4}$ operations with standard resolution-of-the-identity techniques, \cite{Duchemin_2019,Duchemin_2020} this step is the computational bottleneck in the current implementation.

%For comparison purposes, we employ the theoretical best estimates (TBEs) and geometries of Refs.~\onlinecite{Loos_2018a,Loos_2019,Loos_2020b} from which CIS(D), \cite{Head-Gordon_1994,Head-Gordon_1995} ADC(2), \cite{Trofimov_1997,Dreuw_2015} CC2, \cite{Christiansen_1995a} CCSD, \cite{Purvis_1982} and CC3 \cite{Christiansen_1995b} excitation energies are also extracted. 
%Various statistical quantities are reported in the following: the mean signed error (MSE), mean absolute error (MAE), root-mean-square error (RMSE), and the maximum positive [Max($+$)] and maximum negative [Max($-$)] errors.


To conclude, the BSE@$GT$ formalism seems to be, overall, less accurate than BSE@$GW$ in the context of the computation of molecular excitation energies.
However, it still outperforms conventional methods such CIS and TDHF.
It would be interesting to investigate its performance for the computation of ground-state correlation energies within the adiabatic connection fluctuation dissipation formalism where BSE@$GW$ has been shown to be particularly outstanding. \cite{Maggio_2016,Holzer_2018b,Loos_2020e}
The combination of $GT$ and $GW$ via the range separation of the Coulomb operator to avoid double counting of the low-order diagrams is also a promising avenue.
Work along these lines are currently under progress.

%%%%%%%%%%%%%%%%%%%%%%%%
\bibliography{Tmatrix}
%%%%%%%%%%%%%%%%%%%%%%%%

\end{document}
