\documentclass[aip,jcp,reprint,noshowkeys,superscriptaddress]{revtex4-1}
\usepackage{graphicx,dcolumn,bm,xcolor,microtype,multirow,amscd,amsmath,amssymb,amsfonts,physics,longtable,wrapfig,txfonts}
\usepackage[version=4]{mhchem}

\usepackage[utf8]{inputenc}
\usepackage[T1]{fontenc}
\usepackage{txfonts}

\usepackage[
	colorlinks=true,
    citecolor=blue,
    breaklinks=true
	]{hyperref}
\urlstyle{same}

\newcommand{\ie}{\textit{i.e.}}
\newcommand{\eg}{\textit{e.g.}}
\newcommand{\alert}[1]{\textcolor{red}{#1}}
\usepackage[normalem]{ulem}
\newcommand{\titou}[1]{\textcolor{red}{#1}}
\newcommand{\trashPFL}[1]{\textcolor{red}{\sout{#1}}}
\newcommand{\trashXB}[1]{\textcolor{darkgreen}{\sout{#1}}}
\newcommand{\PFL}[1]{\titou{(\underline{\bf PFL}: #1)}}

\newcommand{\mc}{\multicolumn}
\newcommand{\fnm}{\footnotemark}
\newcommand{\fnt}{\footnotetext}
\newcommand{\tabc}[1]{\multicolumn{1}{c}{#1}}
\newcommand{\SI}{\textcolor{blue}{supporting information}}
\newcommand{\QP}{\textsc{quantum package}}
\newcommand{\T}[1]{#1^{\intercal}}

% coordinates
\newcommand{\br}{\mathbf{r}}
\newcommand{\bx}{\mathbf{x}}

% methods
\newcommand{\evGW}{ev$GW$}	
\newcommand{\qsGW}{qs$GW$}	
\newcommand{\GOWO}{$G_0W_0$}	
\newcommand{\Hxc}{\text{Hxc}}
\newcommand{\xc}{\text{xc}}
\newcommand{\Ha}{\text{H}}
\newcommand{\co}{\text{c}}
\newcommand{\x}{\text{x}}

% 
\newcommand{\Norb}{N_\text{orb}}
\newcommand{\Nocc}{O}
\newcommand{\Nvir}{V}

% operators
\newcommand{\hH}{\Hat{H}}
\newcommand{\hS}{\Hat{S}}

% methods
\newcommand{\KS}{\text{KS}}
\newcommand{\HF}{\text{HF}}
\newcommand{\RPA}{\text{RPA}}
\newcommand{\ppRPA}{\text{pp-RPA}}
\newcommand{\BSE}{\text{BSE}}
\newcommand{\dBSE}{\text{dBSE}}
\newcommand{\GW}{GW}
\newcommand{\GT}{GT}
\newcommand{\stat}{\text{stat}}
\newcommand{\dyn}{\text{dyn}}
\newcommand{\TDA}{\text{TDA}}

% energies
\newcommand{\Enuc}{E^\text{nuc}}
\newcommand{\Ec}{E_\text{c}}
\newcommand{\EHF}{E^\text{HF}}
\newcommand{\EBSE}{E^\text{BSE}}
\newcommand{\EcRPA}{E_\text{c}^\text{RPA}}
\newcommand{\EcBSE}{E_\text{c}^\text{BSE}}

% orbital energies
\newcommand{\e}[2]{\eps_{#1}^{#2}}
\newcommand{\eHF}[1]{\eps^\text{HF}_{#1}}
\newcommand{\eKS}[1]{\eps^\text{KS}_{#1}}
\newcommand{\eQP}[1]{\eps^\text{QP}_{#1}}
\newcommand{\eGOWO}[1]{\eps^\text{\GOWO}_{#1}}
\newcommand{\eGW}[1]{\eps^{GW}_{#1}}
\newcommand{\eevGW}[1]{\eps^\text{\evGW}_{#1}}
\newcommand{\eGnWn}[2]{\eps^\text{\GnWn{#2}}_{#1}}
\newcommand{\Om}[2]{\Omega_{#1}^{#2}}
\newcommand{\tOm}[2]{\Tilde{\Omega}_{#1}^{#2}}



% Matrix elements
\newcommand{\tA}[2]{\Tilde{A}_{#1}^{#2}}
\renewcommand{\S}[1]{S_{#1}}
\newcommand{\ABSE}[2]{A_{#1}^{#2,\text{BSE}}}
\newcommand{\BBSE}[2]{B_{#1}^{#2,\text{BSE}}}
\newcommand{\ARPA}[2]{A_{#1}^{#2,\text{RPA}}}
\newcommand{\BRPA}[2]{B_{#1}^{#2,\text{RPA}}}
\newcommand{\ARPAx}[2]{A_{#1}^{#2,\text{RPAx}}}
\newcommand{\BRPAx}[2]{B_{#1}^{#2,\text{RPAx}}}
\newcommand{\G}[1]{G_{#1}}
\newcommand{\LBSE}[1]{L_{#1}}
\newcommand{\XiBSE}[1]{\Xi_{#1}}
\newcommand{\Po}[1]{P_{#1}}
\newcommand{\W}[2]{W_{#1}^{#2}}
\newcommand{\tW}[2]{\widetilde{W}_{#1}^{#2}}
\newcommand{\Wc}[1]{W^\text{c}_{#1}}
\newcommand{\vc}[1]{v_{#1}}
\newcommand{\Sig}[2]{\Sigma_{#1}^{#2}}
\newcommand{\SigC}[1]{\Sigma^\text{c}_{#1}}
\newcommand{\SigX}[1]{\Sigma^\text{x}_{#1}}
\newcommand{\SigXC}[1]{\Sigma^\text{xc}_{#1}}
\newcommand{\Z}[1]{Z_{#1}}
\newcommand{\MO}[1]{\phi_{#1}}
\newcommand{\SO}[1]{\psi_{#1}}
\newcommand{\ERI}[2]{(#1|#2)}
\newcommand{\rbra}[1]{(#1|}
\newcommand{\rket}[1]{|#1)}
\newcommand{\sERI}[2]{[#1|#2]}

%% bold in Table
\newcommand{\bb}[1]{\textbf{#1}}
\newcommand{\rb}[1]{\textbf{\textcolor{red}{#1}}}
\newcommand{\gb}[1]{\textbf{\textcolor{darkgreen}{#1}}}

% excitation energies
\newcommand{\OmRPA}[1]{\Omega_{#1}^{\text{RPA}}}
\newcommand{\OmRPAx}[1]{\Omega_{#1}^{\text{RPAx}}}
\newcommand{\OmBSE}[1]{\Omega_{#1}^{\text{BSE}}}


% Matrices
\newcommand{\bO}{\mathbf{0}}
\newcommand{\bI}{\mathbf{1}}
\newcommand{\bvc}{\mathbf{v}}
\newcommand{\bSig}{\mathbf{\Sigma}}
\newcommand{\bSigX}{\mathbf{\Sigma}^\text{x}}
\newcommand{\bSigC}{\mathbf{\Sigma}^\text{c}}
\newcommand{\bSigGW}{\mathbf{\Sigma}^{GW}}
\newcommand{\be}{\mathbf{\epsilon}}
\newcommand{\beGW}{\mathbf{\epsilon}^{GW}}
\newcommand{\beGnWn}[1]{\mathbf{\epsilon}^\text{\GnWn{#1}}}
\newcommand{\bde}{\mathbf{\Delta\epsilon}}
\newcommand{\bdeHF}{\mathbf{\Delta\epsilon}^\text{HF}}
\newcommand{\bdeGW}{\mathbf{\Delta\epsilon}^{GW}}
\newcommand{\bOm}[1]{\mathbf{\Omega}^{#1}}
\newcommand{\bA}[2]{\mathbf{A}_{#1}^{#2}}
\newcommand{\bB}[2]{\mathbf{B}_{#1}^{#2}}
\newcommand{\bC}[2]{\mathbf{C}_{#1}^{#2}}
\newcommand{\bX}[2]{\mathbf{X}_{#1}^{#2}}
\newcommand{\bY}[2]{\mathbf{Y}_{#1}^{#2}}
\newcommand{\bZ}[2]{\mathbf{Z}_{#1}^{#2}}
\newcommand{\bK}{\mathbf{K}}
\newcommand{\bP}[1]{\mathbf{P}^{#1}}

% units
\newcommand{\IneV}[1]{#1 eV}
\newcommand{\InAU}[1]{#1 a.u.}
\newcommand{\InAA}[1]{#1 \AA}
\newcommand{\kcal}{kcal/mol}

% orbitals, gaps, etc
\newcommand{\eps}{\varepsilon}
\newcommand{\IP}{I}
\newcommand{\EA}{A}
\newcommand{\HOMO}{\text{HOMO}}
\newcommand{\LUMO}{\text{LUMO}}
\newcommand{\Eg}{E_\text{g}}
\newcommand{\EgFun}{\Eg^\text{fund}}
\newcommand{\EgOpt}{\Eg^\text{opt}}
\newcommand{\EB}{E_B}

\newcommand{\sig}{\sigma}
\newcommand{\bsig}{{\Bar{\sigma}}}
\newcommand{\sigp}{{\sigma'}}
\newcommand{\bsigp}{{\Bar{\sigma}'}}
\newcommand{\taup}{{\tau'}}

\newcommand{\up}{\uparrow}
\newcommand{\dw}{\downarrow}
\newcommand{\upup}{\uparrow\uparrow}
\newcommand{\updw}{\uparrow\downarrow}
\newcommand{\dwup}{\downarrow\uparrow}
\newcommand{\dwdw}{\downarrow\downarrow}
\newcommand{\spc}{\text{sc}}
\newcommand{\spf}{\text{sf}}
\newcommand{\ssp}{\text{ss}}
\newcommand{\osp}{\text{os}}

% addresses
\newcommand{\LCPQ}{Laboratoire de Chimie et Physique Quantiques (UMR 5626), Universit\'e de Toulouse, CNRS, UPS, France}
\newcommand{\LPT}{Laboratoire de Physique Th\'eorique, Universit\'e de Toulouse, CNRS, UPS, France}
\newcommand{\ETSF}{European Theoretical Spectroscopy Facility (ETSF)}
\begin{document}	

\title{Static and Dynamic Bethe-Salpeter Equations in the $T$-Matrix Approximation}
\author{Pierre-Fran\c{c}ois \surname{Loos}}
	\email{loos@irsamc.ups-tlse.fr}
	\affiliation{\LCPQ}
\author{Pina Romaniello}
	\email{romaniello@irsamc.ups-tlse.fr}
	\affiliation{\LPT}
	\affiliation{\ETSF}

\begin{abstract}
We report, for the first time, the static and dynamic Bethe-Salpeter equations (BSE) within the so-called $T$-matrix approximation (BSE@$GT$).
The performance of the (static) BSE@$GT$ scheme and its perturbative dynamical correction (dBSE@$GT$) are assessed by computed the neutral excited states of molecular systems.
%\bigskip
%\begin{center}
%	\boxed{\includegraphics[width=0.5\linewidth]{TOC}}
%\end{center}
%\bigskip
\end{abstract}

\maketitle

The $GW$ approximation \cite{Hedin_1965} is an elegant resummation of all direct ring diagrams from the particle-hole (ph) channel and is justified in the high-density or weakly-correlated regime. 
Within the $GW$ approximation, a key quantity of many-body perturbation theory, the self-energy $\Sigma$, reads
\begin{equation}
	\Sigma^{GW}(1,2) = i G(1,2) W(1,2)
\end{equation}
where $G(1,2)$ is the one-body Green's function and $W(1,2)$ is the dynamically-screened Coulomb potential, while, \eg, $1 \equiv (\sigma_1,\br_1,t_1)$ is a composite coordinate gathering spin, space, and time variables.

Alternatives to $GW$ do exist. For example, the $T$-matrix (or Bethe-Goldstone) approximation, first introduced in nuclear physics and then in condensed matter physics, sums the ladder diagrams from the particle-particle (pp) channel
and is justified in the low-density or strongly-correlated regime.
While $GW$ considers a two-point interaction, the $T$-matrix approximation relies on a more complex (four-point) effective interaction --- the so-called $T$ matrix --- yielding the following self-energy: 
\begin{equation}
	\Sigma^{GT}(1,2) = i \int G(4,3) T(1,3;2,4) d3 d4
\end{equation}
Combining these two channels, which sounds like a natural thing to do, is also possible and has been explored in the Hubbard dimer. \cite{Romaniello_2012}

Let us consider a closed-shell electronic system consisting of $N$ electrons and $K$ one-electron basis functions.
The number of singly-occupied and virtual (\ie, unoccupied) spinorbitals are $O = N$ and $V = K - O$, respectively.
Let us denote as $\SO{p}(\bx)$ the $p$th spinorbital and $\e{p}{}$ its one-electron energy.
The composite variable $\bx = (\sigma,\br)$ gathers spin ($\sigma$) and spatial ($\br$) variables.
We assume real quantities throughout this manuscript, $i$, $j$, $k$, and $l$ are occupied orbitals, $a$, $b$, $c$, and $d$ are unoccupied orbitals, $p$, $q$, $r$, and $s$ indicate arbitrary orbitals, $m$ labels single excitations, while $n$ labels double electron attachments or double electron detachments.

By definition, the one-body Green's function is \cite{Martin_2016}
\begin{equation}
\label{eq:G}
	G(\bx_1,\bx_2;\omega) 
	= \sum_i \frac{\SO{i}(\bx_1) \SO{i}(\bx_2)}{\omega - \e{i}{} - i\eta}	
	+ \sum_a \frac{\SO{a}(\bx_1) \SO{a}(\bx_2)}{\omega - \e{a}{} + i\eta}	
\end{equation}
where $\eta$ is a positive infinitesimal, and its nature is completely defined by the set of orbitals and corresponding energies that are used to build it.
For example, $G^{\HF}(\bx_1,\bx_2;\omega)$ is the Hartree-Fock (HF) Green's function built from the HF spinorbitals $\SO{p}^{\HF}(\bx)$ and energies $\e{p}{\HF}$.

Contrary to the $GW$ approximation which relies on the (two-point) dynamically-screened Coulomb potential $W$ computed from a ph-random-phase approximation (ph-RPA) problem to target charged excitations (\ie, ionization potentials and electron affinities), \cite{Hedin_1965,Aryasetiawan_1998,Onida_2002,Martin_2016,Reining_2017,Golze_2019} here we consider the $GT$ approximation where one employs the (four-point) $T$ matrix obtained from solving the pp-RPA equations.
The non-Hermitian pp-RPA problem reads \cite{Schuck_Book,vanAggelen_2013,Peng_2013,Scuseria_2013,Yang_2013,Yang_2013b,Yang_2014a}
\begin{equation}
\label{eq:LR-RPA}
	\begin{pmatrix}
		\bA{}{\ppRPA}			&	\bB{}{\ppRPA}	\\
		-\T{(\bB{}{\ppRPA})}	&	-\bC{}{\ppRPA}	\\
	\end{pmatrix}
	\cdot
	\begin{pmatrix}
		\bX{n}{N\pm2}	\\
		\bY{n}{N\pm2}	\\
	\end{pmatrix}
	=
	\Om{n}{N\pm2}
	\begin{pmatrix}
		\bX{n}{N\pm2}	\\
		\bY{n}{N\pm2}	\\
	\end{pmatrix}
\end{equation}
where the elements of the various matrices are defined as
\begin{subequations}
\begin{align}
	A_{ab,cd}^{\ppRPA} & = \delta_{ab} \delta_{cd} (\e{a}{} + \e{b}{}) + \mel{ab}{}{cd}
	\\ 
	B_{ab,ij}^{\ppRPA} & = \mel{ab}{}{ij}
	\\ 
	C_{ij,kl}^{\ppRPA} & = - \delta_{ik} \delta_{jl} (\e{i}{} + \e{j}{}) +\mel{ij}{}{kl}
\end{align}
\end{subequations}
and 
\begin{equation}
	\mel{pq}{}{rs} = \braket{pq}{rs} - \braket{pq}{sr}
\end{equation}
are two-electron integrals in the spinorbital basis, \ie,
\begin{equation}
	\braket{pq}{rs} = \iint \SO{p}(\bx_1) \SO{q}(\bx_1) \frac{1}{\abs{\br_1 - \br_2}} \SO{r}(\bx_2) \SO{s}(\bx_2)  d\bx_1 d\bx_2
\end{equation}
The pp-RPA problem yields, in the absence of instabilities, $V(V-1)/2$ positive eigenvalues $\Om{n}{N+2}$ and $O(O-1)/2$ negative eigenvalues $\Om{n}{N-2}$, which  correspond respectively to double attachments and double detachments.
The pp-RPA correlation energies is given by \cite{Peng_2013,Scuseria_2013}
\begin{equation}
\begin{split}
	\Ec^\ppRPA 
	& = + \sum_n \Om{n}{N+2} - \Tr(\bA{}{\ppRPA}) 
	\\
	& = - \sum_n \Om{n}{N-2} - \Tr(\bC{}{\ppRPA})
\end{split}
\end{equation}

Based on the knowledge of the pp-RPA eigenvalues and eigenvectors, one can then construct the $T$ matrix.
In the spinorbital basis, the correlation part of the $T$ matrix has the following expression
\begin{equation}
\label{eq:T}
	T_{pq,rs}(\omega) 
		= \sum_{n} \frac{\braket*{pq}{\chi_n^{N+2}}\braket*{rs}{\chi_n^{N+2}}}{\omega - \Om{n}{N+2} + i\eta}
		- \sum_{n} \frac{\braket*{pq}{\chi_n^{N-2}}\braket*{rs}{\chi_n^{N-2}}}{\omega - \Om{n}{N-2} - i\eta}
\end{equation}
with
\begin{subequations}
\begin{align}
	\braket*{pq}{\chi_n^{N+2}} & = \sum_{c < d} \mel{pq}{}{cd} X_{cd,n}^{N+2} + \sum_{k < l}  \mel{pq}{}{kl} Y_{kl,n}^{N+2}
	\\
	\braket*{pq}{\chi_n^{N-2}} & = \sum_{c < d} \mel{pq}{}{cd} X_{cd,n}^{N-2} + \sum_{k < l}  \mel{pq}{}{kl} X_{kl,n}^{N-2}
\end{align}
\end{subequations}

Combining Eqs.~\eqref{eq:G} and \eqref{eq:T}, the correlation part of the $T$-matrix self-energy reads \cite{Romaniello_2012,Martin_2016,Zhang_2017,Li_2021b}
\begin{equation}
	\Sigma^{GT}_{pq}(\omega)
	= \sum_{in} \frac{\braket*{pi}{\chi_n^{N+2}}\braket*{qi}{\chi_n^{N+2}}}{\omega + \e{i}{} - \Om{n}{N+2} + i\eta}
	+ \sum_{an} \frac{\braket*{pa}{\chi_n^{N-2}}\braket*{qa}{\chi_n^{N-2}}}{\omega + \e{a}{} - \Om{n}{N-2} - i\eta}
\end{equation}

Within the (perturbative) one-shot $GT$ scheme (labeled as $G_0T_0$ in the following), the quasiparticle energies are obtained via linearization of the quasiparticle equation, \ie,
\begin{equation}
	\e{p}{G_0T_0} = \e{p}{\HF} + Z_p \Sigma^{GT}_{pp}(\e{p}{\HF})
\end{equation}
where we have assumed a HF starting point and
\begin{equation}
	Z_p = \qty[ 1 - \eval{\pdv{\Sigma^{T}_{pq}(\omega)}{\omega}}_{\omega = \e{p}{\HF}} ]^{-1}
\end{equation}
is the renormalization factor.
Other levels of self-consistency can be considered like the partially self-consistent \textit{``eigenvalue''} $GT$ (ev$GT$) or the quasiparticle self-consistent $GT$ (qs$GT$) scheme.

In order to target neutral (or optical) optical excitations, we now consider the static version of the Bethe-Salpeter equation (BSE) \cite{Salpeter_1951,Strinati_1988,Blase_2018,Blase_2020} employing the $GT$ quasiparticle energies as well as $T$-matrix kernel.
In this case, the BSE@$GT$ linear eigenvalue problem reads
\begin{equation}
\label{eq:BSE}
	\begin{pmatrix}
		\bA{}{\BSE}		&	\bB{}{\BSE}	\\
		-\bB{}{\BSE}	&	-\bA{}{\BSE}	\\
	\end{pmatrix}
	\cdot
	\begin{pmatrix}
		\bX{m}{\BSE}	\\
		\bY{m}{\BSE}	\\
	\end{pmatrix}
	=
	\Om{m}{\BSE}
	\begin{pmatrix}
		\bX{m}{\BSE}	\\
		\bY{m}{\BSE}	\\
	\end{pmatrix}
\end{equation}
with 
\begin{subequations}
\begin{align}
	A_{ia,jb}^{\BSE} & = \delta_{ij} \delta_{ab} (\e{a}{\GT} + \e{i}{\GT}) + \mel{ib}{}{aj} + T_{ij,ba}(\omega=0)
	\\ 
	B_{ia,jb}^{\BSE} & = \mel{ij}{}{ab} + T_{ib,ja}(\omega=0)
\end{align}
\end{subequations}
The eigenvalues of Eq.~\eqref{eq:BSE} provide $OV$ singlet and $OV$ triplet single excitations.
Due to the frequency-independent nature of static BSE equations , it is well known that one cannot access higher excitations.

In order to go beyond the static approximation, it is possible to consider the dynamical version of the BSE (dBSE). \cite{Strinati_1988}
In this case, one must solve the (non-linear) dynamical eigenvalue problem
\begin{equation}
\label{eq:LR-RPA}
	\begin{pmatrix}
		\bA{}{\dBSE}(\Om{S}{\dBSE})	&	\bB{}{\BSE}	\\
		-\bB{}{\BSE}	&	-\bA{}{\dBSE}(-\Om{S}{\dBSE})	\\
	\end{pmatrix}
	\cdot
	\begin{pmatrix}
		\bX{S}{\dBSE}	\\
		\bY{S}{\dBSE}	\\
	\end{pmatrix}
	=
	\Om{S}{\dBSE}
	\begin{pmatrix}
		\bX{S}{\dBSE}	\\
		\bY{S}{\dBSE}	\\
	\end{pmatrix}
\end{equation}
with 
\begin{equation}
	A_{ia,jb}^{\dBSE}(\omega) = \delta_{ij} \delta_{ab} (\e{a}{\GT} + \e{i}{\GT}) + \mel{ib}{}{aj} + \widetilde{T}_{ij,ba}(\omega)
\end{equation}
where, following Strinati, \cite{Strinati_1988} one can derive the following expression for the elements of the dynamical $T$ matrix
\begin{equation}
\label{eq:dT}
\begin{split}
	\widetilde{T}_{pq,rs}(\omega) 
		& = \sum_{n} \frac{\braket*{pq}{\chi_n^{N+2}}\braket*{rs}{\chi_n^{N+2}}}{\omega - \Om{n}{N+2} + (\e{i}{\GT} + \e{j}{\GT}) + i\eta}
		\\
		& + \sum_{n} \frac{\braket*{pq}{\chi_n^{N-2}}\braket*{rs}{\chi_n^{N-2}}}{\omega + \Om{n}{N-2} - (\e{a}{\GT} + \e{b}{\GT}) + i\eta}
\end{split}
\end{equation}
Equation \eqref{eq:dT} highlights the interesting dynamical structure of the $T$ matrix, where, similarly to the dBSE@$GW$ scheme, the 2h2p configurations are downfolded on the 1h1p configurations.
Because solving a non-linear eigenvalue problem is computationally challenging, here we rely on the dynamical perturbative scheme within the Tamm-Dancoff approximation.
\cite{Rohlfing_2000,Romaniello_2009b,Ma_2009a,Ma_2009b,Zhang_2013,Rebolini_2016,Olevano_2019,Loos_2020h,Authier_2020,Monino_2021}
%lthough one can potentially access double excitations within this formalism (due to the dynamical nature of the equations), here we rely on the perturbative scheme developed in Ref.~\onlinecite{Loos_2020h} in order to access corrected single excitations.

The present formalism has been implemented in QuAcK \cite{QuAcK} and we consider here only systems with closed-shell singlet ground states.
The $GW$ and $GT$ calculations are performed by considering a (restricted) HF starting point and standard gaussian basis sets (defined with cartesian functions) are considered.


%%%%%%%%%%%%%%%%%%%%%%%%
\bibliography{Tmatrix}
%%%%%%%%%%%%%%%%%%%%%%%%

\end{document}
