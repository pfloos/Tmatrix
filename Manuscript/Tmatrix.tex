\documentclass[aip,jcp,reprint,noshowkeys,superscriptaddress]{revtex4-1}
\usepackage{graphicx,dcolumn,bm,xcolor,microtype,multirow,amscd,amsmath,amssymb,amsfonts,physics,longtable,wrapfig,txfonts}
\usepackage[version=4]{mhchem}

\usepackage[utf8]{inputenc}
\usepackage[T1]{fontenc}
\usepackage{txfonts}

\usepackage[
	colorlinks=true,
    citecolor=blue,
    breaklinks=true
	]{hyperref}
\urlstyle{same}

\newcommand{\ie}{\textit{i.e.}}
\newcommand{\eg}{\textit{e.g.}}
\newcommand{\alert}[1]{\textcolor{red}{#1}}
\usepackage[normalem]{ulem}
\newcommand{\titou}[1]{\textcolor{red}{#1}}
\newcommand{\trashPFL}[1]{\textcolor{red}{\sout{#1}}}
\newcommand{\trashXB}[1]{\textcolor{darkgreen}{\sout{#1}}}
\newcommand{\PFL}[1]{\titou{(\underline{\bf PFL}: #1)}}

\newcommand{\mc}{\multicolumn}
\newcommand{\fnm}{\footnotemark}
\newcommand{\fnt}{\footnotetext}
\newcommand{\tabc}[1]{\multicolumn{1}{c}{#1}}
\newcommand{\SI}{\textcolor{blue}{supporting information}}
\newcommand{\QP}{\textsc{quantum package}}
\newcommand{\T}[1]{#1^{\intercal}}

% coordinates
\newcommand{\br}{\mathbf{r}}
\newcommand{\bx}{\mathbf{x}}
\newcommand{\dbr}{d\br}
\newcommand{\dbx}{d\bx}

% methods
\newcommand{\evGW}{ev$GW$}	
\newcommand{\qsGW}{qs$GW$}	
\newcommand{\GOWO}{$G_0W_0$}	
\newcommand{\Hxc}{\text{Hxc}}
\newcommand{\xc}{\text{xc}}
\newcommand{\Ha}{\text{H}}
\newcommand{\co}{\text{c}}
\newcommand{\x}{\text{x}}

% 
\newcommand{\Norb}{N_\text{orb}}
\newcommand{\Nocc}{O}
\newcommand{\Nvir}{V}

% operators
\newcommand{\hH}{\Hat{H}}
\newcommand{\hS}{\Hat{S}}

% methods
\newcommand{\KS}{\text{KS}}
\newcommand{\HF}{\text{HF}}
\newcommand{\RPA}{\text{RPA}}
\newcommand{\ppRPA}{\text{pp-RPA}}
\newcommand{\BSE}{\text{BSE}}
\newcommand{\dBSE}{\text{dBSE}}
\newcommand{\GW}{GW}
\newcommand{\stat}{\text{stat}}
\newcommand{\dyn}{\text{dyn}}
\newcommand{\TDA}{\text{TDA}}

% energies
\newcommand{\Enuc}{E^\text{nuc}}
\newcommand{\Ec}{E_\text{c}}
\newcommand{\EHF}{E^\text{HF}}
\newcommand{\EBSE}{E^\text{BSE}}
\newcommand{\EcRPA}{E_\text{c}^\text{RPA}}
\newcommand{\EcBSE}{E_\text{c}^\text{BSE}}

% orbital energies
\newcommand{\e}[1]{\eps_{#1}}
\newcommand{\eHF}[1]{\eps^\text{HF}_{#1}}
\newcommand{\eKS}[1]{\eps^\text{KS}_{#1}}
\newcommand{\eQP}[1]{\eps^\text{QP}_{#1}}
\newcommand{\eGOWO}[1]{\eps^\text{\GOWO}_{#1}}
\newcommand{\eGW}[1]{\eps^{GW}_{#1}}
\newcommand{\eevGW}[1]{\eps^\text{\evGW}_{#1}}
\newcommand{\eGnWn}[2]{\eps^\text{\GnWn{#2}}_{#1}}
\newcommand{\Om}[2]{\Omega_{#1}^{#2}}
\newcommand{\tOm}[2]{\Tilde{\Omega}_{#1}^{#2}}



% Matrix elements
\newcommand{\tA}[2]{\Tilde{A}_{#1}^{#2}}
\renewcommand{\S}[1]{S_{#1}}
\newcommand{\ABSE}[2]{A_{#1}^{#2,\text{BSE}}}
\newcommand{\BBSE}[2]{B_{#1}^{#2,\text{BSE}}}
\newcommand{\ARPA}[2]{A_{#1}^{#2,\text{RPA}}}
\newcommand{\BRPA}[2]{B_{#1}^{#2,\text{RPA}}}
\newcommand{\ARPAx}[2]{A_{#1}^{#2,\text{RPAx}}}
\newcommand{\BRPAx}[2]{B_{#1}^{#2,\text{RPAx}}}
\newcommand{\G}[1]{G_{#1}}
\newcommand{\LBSE}[1]{L_{#1}}
\newcommand{\XiBSE}[1]{\Xi_{#1}}
\newcommand{\Po}[1]{P_{#1}}
\newcommand{\W}[2]{W_{#1}^{#2}}
\newcommand{\tW}[2]{\widetilde{W}_{#1}^{#2}}
\newcommand{\Wc}[1]{W^\text{c}_{#1}}
\newcommand{\vc}[1]{v_{#1}}
\newcommand{\Sig}[2]{\Sigma_{#1}^{#2}}
\newcommand{\SigC}[1]{\Sigma^\text{c}_{#1}}
\newcommand{\SigX}[1]{\Sigma^\text{x}_{#1}}
\newcommand{\SigXC}[1]{\Sigma^\text{xc}_{#1}}
\newcommand{\Z}[1]{Z_{#1}}
\newcommand{\MO}[1]{\phi_{#1}}
\newcommand{\SO}[1]{\psi_{#1}}
\newcommand{\ERI}[2]{(#1|#2)}
\newcommand{\rbra}[1]{(#1|}
\newcommand{\rket}[1]{|#1)}
\newcommand{\sERI}[2]{[#1|#2]}

%% bold in Table
\newcommand{\bb}[1]{\textbf{#1}}
\newcommand{\rb}[1]{\textbf{\textcolor{red}{#1}}}
\newcommand{\gb}[1]{\textbf{\textcolor{darkgreen}{#1}}}

% excitation energies
\newcommand{\OmRPA}[1]{\Omega_{#1}^{\text{RPA}}}
\newcommand{\OmRPAx}[1]{\Omega_{#1}^{\text{RPAx}}}
\newcommand{\OmBSE}[1]{\Omega_{#1}^{\text{BSE}}}


% Matrices
\newcommand{\bO}{\mathbf{0}}
\newcommand{\bI}{\mathbf{1}}
\newcommand{\bvc}{\mathbf{v}}
\newcommand{\bSig}{\mathbf{\Sigma}}
\newcommand{\bSigX}{\mathbf{\Sigma}^\text{x}}
\newcommand{\bSigC}{\mathbf{\Sigma}^\text{c}}
\newcommand{\bSigGW}{\mathbf{\Sigma}^{GW}}
\newcommand{\be}{\mathbf{\epsilon}}
\newcommand{\beGW}{\mathbf{\epsilon}^{GW}}
\newcommand{\beGnWn}[1]{\mathbf{\epsilon}^\text{\GnWn{#1}}}
\newcommand{\bde}{\mathbf{\Delta\epsilon}}
\newcommand{\bdeHF}{\mathbf{\Delta\epsilon}^\text{HF}}
\newcommand{\bdeGW}{\mathbf{\Delta\epsilon}^{GW}}
\newcommand{\bOm}[1]{\mathbf{\Omega}^{#1}}
\newcommand{\bA}[2]{\mathbf{A}_{#1}^{#2}}
\newcommand{\bB}[2]{\mathbf{B}_{#1}^{#2}}
\newcommand{\bC}[2]{\mathbf{C}_{#1}^{#2}}
\newcommand{\bX}[2]{\mathbf{X}_{#1}^{#2}}
\newcommand{\bY}[2]{\mathbf{Y}_{#1}^{#2}}
\newcommand{\bZ}[2]{\mathbf{Z}_{#1}^{#2}}
\newcommand{\bK}{\mathbf{K}}
\newcommand{\bP}[1]{\mathbf{P}^{#1}}

% units
\newcommand{\IneV}[1]{#1 eV}
\newcommand{\InAU}[1]{#1 a.u.}
\newcommand{\InAA}[1]{#1 \AA}
\newcommand{\kcal}{kcal/mol}

% orbitals, gaps, etc
\newcommand{\eps}{\varepsilon}
\newcommand{\IP}{I}
\newcommand{\EA}{A}
\newcommand{\HOMO}{\text{HOMO}}
\newcommand{\LUMO}{\text{LUMO}}
\newcommand{\Eg}{E_\text{g}}
\newcommand{\EgFun}{\Eg^\text{fund}}
\newcommand{\EgOpt}{\Eg^\text{opt}}
\newcommand{\EB}{E_B}

\newcommand{\sig}{\sigma}
\newcommand{\bsig}{{\Bar{\sigma}}}
\newcommand{\sigp}{{\sigma'}}
\newcommand{\bsigp}{{\Bar{\sigma}'}}
\newcommand{\taup}{{\tau'}}

\newcommand{\up}{\uparrow}
\newcommand{\dw}{\downarrow}
\newcommand{\upup}{\uparrow\uparrow}
\newcommand{\updw}{\uparrow\downarrow}
\newcommand{\dwup}{\downarrow\uparrow}
\newcommand{\dwdw}{\downarrow\downarrow}
\newcommand{\spc}{\text{sc}}
\newcommand{\spf}{\text{sf}}
\newcommand{\ssp}{\text{ss}}
\newcommand{\osp}{\text{os}}

% addresses
\newcommand{\LCPQ}{Laboratoire de Chimie et Physique Quantiques (UMR 5626), Universit\'e de Toulouse, CNRS, UPS, France}

\begin{document}	

\title{Spin structure of the T-matrix}

\author{Pierre-Fran\c{c}ois \surname{Loos}}
	\email{loos@irsamc.ups-tlse.fr}
	\affiliation{\LCPQ}

\begin{abstract}
%\bigskip
%\begin{center}
%	\boxed{\includegraphics[width=0.5\linewidth]{TOC}}
%\end{center}
%\bigskip
\end{abstract}

\maketitle

%%%%%%%%%%%%%%%%%%%%%%%%%%%%%%%%%%%%%%%%%%%%%%%
\section{Unrestricted $GT$ formalism}
\label{sec:UGW}
%%%%%%%%%%%%%%%%%%%%%%%%%%%%%%%%%%%%%%%%%%%%%%%
Let us consider an electronic system consisting of $N = N_\up + N_\dw$ electrons (where $N_\up$ and $N_\dw$ are the number of spin-up and spin-down electrons, respectively) and $K$ one-electron basis functions.
The number of spin-up and spin-down occupied orbitals are $O_\up = N_\up$ and $O_\dw = N_\dw$, respectively, and, assuming the absence of linear dependencies in the one-electron basis set, there is $V_\up = K - O_\up$ and $V_\dw = K - O_\dw$ spin-up and spin-down virtual (\ie, unoccupied) orbitals.
%The number of spin-conserved (sc) single excitations is then $S^\spc = S_{\up\up}^\spc + S_{\dw\dw}^\spc = O_\up V_\up + O_\dw V_\dw$, while the number of spin-flip (sf) excitations is $S^\spf = S_{\up\dw}^\spf + S_{\dw\up}^\spf = O_\up V_\dw + O_\dw V_\up$.
Let us denote as $\MO{p_\sig}(\br)$ the $p$th spatial orbital associated with the spin-$\sig$ electrons (where $\sig =$ $\up$ or $\dw$) and $\e{p_\sig}{}$ its one-electron energy.
%It is important to understand that, in a spin-conserved excitation the hole orbital $\MO{i_\sig}$ and particle orbital $\MO{a_\sig}$ have the same spin $\sig$.
%In a spin-flip excitation, the hole and particle states, $\MO{i_\sig}$ and $\MO{a_\bsig}$, have opposite spins, $\sig$ and $\bsig$.
We assume real quantities throughout this manuscript, $i$ and $j$ are occupied orbitals, $a$ and $b$ are unoccupied orbitals, $p$, $q$, $r$, and $s$ indicate arbitrary orbitals, and $m$ labels double electron attachment or double electron detachment.

In a pp-RPA calculation, one considers double ionizations and double electron attachments.
However, these processes can both involved a pair of same-spin (ss) electrons or opposite-spin (os) electrons.
For the particle-particle states, we have
\begin{align}
	P^{\osp} & = V_\up V_\dw
	\\
	P^{\ssp} & = P_{\up\up}^{\ssp} + P_{\dw\dw}^{\ssp}
	&
	P_{\up\up}^{\ssp} & = V_\up(V_\up - 1)/2
	&
	P_{\dw\dw}^{\ssp} & = V_\dw(V_\dw - 1)/2
\end{align}
For the hole-hole states, we have
\begin{align}
	H^{\osp} & = O_\up O_\dw
	\\
	H^{\ssp} & = H_{\up\up}^{\ssp} + H_{\dw\dw}^{\ssp}
	&
	H_{\up\up}^{\ssp} & = O_\up(O_\up - 1)/2
	&
	H_{\dw\dw}^{\ssp} & = O_\dw(O_\dw - 1)/2
\end{align}

%==================================================
\subsection{Spin structure of the Green's function}
%==================================================

By definition, the one-body Green's function is
\begin{equation}
\label{eq:G}
	G(\bx_1,\bx_2;\omega) 
	= \sum_i \frac{\SO{i}(\bx_1) \SO{i}(\bx_2)}{\omega - \e{i}{} - i\eta}	
	+ \sum_a \frac{\SO{a}(\bx_1) \SO{a}(\bx_2)}{\omega - \e{a}{} + i\eta}	
\end{equation}
where $\eta$ is a positive infinitesimal and $\SO{p}(\bx)$ is a spinorbital.
A sum over spinorbitals can always be decomposed as such $\sum_p = \sum_{p_\sig} + \sum_{p_\bsig}$.
In the spatial orbital basis, the spin-$\sig$ component of the one-body Green's function reads \cite{ReiningBook,Bruneval_2016}
\begin{equation}
\label{eq:G_sig}
	G^{\sig}(\br_1,\br_2;\omega) 
	= \sum_{i} \frac{\MO{i_\sig}(\br_1) \MO{i_\sig}(\br_2)}{\omega - \e{i_\sig}{} - i\eta}	
	+ \sum_{a} \frac{\MO{a_\sig}(\br_1) \MO{a_\sig}(\br_2)}{\omega - \e{a_\sig}{} + i\eta}	
\end{equation}

%========================================================
\subsection{Spin structure of the $T$-matrix}
%========================================================

%========================================================
\subsection{Spin structure of pp-RPA problem}
%========================================================


\begin{equation}
\label{eq:LR-RPA}
	\begin{pmatrix}
		\bA{}{}				&	\bB{}{}	\\
		\bB{}{\intercal}	&	\bC{}{}	\\
	\end{pmatrix}
	\cdot
	\begin{pmatrix}
		\bX{m}{N\pm2}	\\
		\bY{m}{N\pm2}	\\
	\end{pmatrix}
	=
	\Om{m}{N\pm2}
	\begin{pmatrix}
		\bX{m}{N\pm2}	\\
		\bY{m}{N\pm2}	\\
	\end{pmatrix}
\end{equation}
where
\begin{subequations}
\begin{align}
	A_{ab,cd}^{\ppRPA} & = \delta_{ab} \delta_{cd} (\e{a} + \e{b}) + \mel{ab}{}{cd}
	\\ 
	B_{ab,ij}^{\ppRPA} & = \mel{ab}{}{ij}
	\\ 
	C_{ij,kl}^{\ppRPA} & = - \delta_{ik} \delta_{jl} (\e{i} + \e{j}) +\mel{ij}{}{kl}
\end{align}
\end{subequations}
with
\begin{equation}
	\mel{pq}{}{rs} = \braket{pq}{rs} - \braket{pq}{sr}
\end{equation}
The pp-RPA problem yields, in general, $P^{\osp} + P^{\ssp}$ positive eigenvalues $\Om{n}{N+2}$ and $H^{\osp} + H^{\ssp}$ negative eigenvalues $\Om{n}{N-2}$, which  correspond respectively to the double attachments and double detachments.
The present formula can be explicitly given for the different spin blocks ($\up\up\up\up$, $\up\dw\up\dw$, and $\dw\dw\dw\dw$)
\begin{subequations}
\begin{align}
	A_{a_\sig b_\sig,c_\sig d_\sig}^{\ppRPA} & = \delta_{ab} \delta_{cd} (\e{a_\sig} + \e{b_\sig}) + \mel{a_\sig b_\sig}{}{c_\sig d_\sig}
	\\ 
	B_{a_\sig b_\sig,i_\sig j_\sig}^{\ppRPA} & = \mel{a_\sig b_\sig }{}{i_\sig j_\sig}
	\\ 
	C_{i_\sig j_\sig,k_\sig l_\sig}^{\ppRPA} & = - \delta_{ik} \delta_{jl} (\e{i_\sig} + \e{j_\sig}) + \mel{i_\sig j_\sig}{}{k_\sig l_\sig}
\end{align}
\end{subequations}

\begin{subequations}
\begin{align}
	A_{a_\sig b_\bsig,c_\sig d_\bsig}^{\ppRPA} & = \delta_{ab} \delta_{cd} (\e{a_\sig} + \e{b_\bsig}) + \braket{a_\sig b_\bsig}{c_\sig d_\bsig}
	\\ 
	B_{a_\sig b_\bsig,i_\sig j_\bsig}^{\ppRPA} & = \braket{a_\sig b_\bsig }{i_\sig j_\bsig}
	\\ 
	C_{i_\sig j_\bsig,k_\sig l_\bsig}^{\ppRPA} & = - \delta_{ik} \delta_{jl} (\e{i_\sig} + \e{j_\bsig}) + \braket{i_\sig j_\bsig}{k_\sig l_\bsig}
\end{align}
\end{subequations}


%\begin{subequations}
%\begin{align}
%\label{eq:LR-RPA-AB-sc}
%	\bA{}{\spc} & = \begin{pmatrix}
%		\bA{}{\upup,\upup}	&	\bA{}{\upup,\dwdw}	\\
%		\bA{}{\dwdw,\upup}	&	\bA{}{\dwdw,\dwdw}	\\
%	\end{pmatrix}
%	&
%	\bB{}{\spc} & = \begin{pmatrix}
%		\bB{}{\upup,\upup}	&	\bB{}{\upup,\dwdw}	\\
%		\bB{}{\dwdw,\upup}	&	\bB{}{\dwdw,\dwdw}	\\
%	\end{pmatrix}
%\\
%\label{eq:LR-RPA-AB-sf}
%	\bA{}{\spf} & = \begin{pmatrix}
%		\bA{}{\updw,\updw}	&	\bO	\\
%		\bO					&	\bA{}{\dwup,\dwup}	\\
%	\end{pmatrix}
%	&
%	\bB{}{\spf} & = \begin{pmatrix}
%		\bO					&	\bB{}{\updw,\dwup}	\\
%		\bB{}{\dwup,\updw}	&	\bO	\\
%	\end{pmatrix}
%\end{align}
%\end{subequations}

%========================================================
\subsection{Spin structure of the $T$-matrix}
%========================================================
According to Roberto, in the spinorbital basis, the correlation part of the $T$-matrix has the following expression
\begin{equation}
\begin{split}
	T_{pqrs}(\omega) 
		& = T_{pqrs}^{N+2}(\omega) + T_{pqrs}^{N-2}(\omega) 
		\\
		& = \sum_{m} \frac{\braket{pq}{\chi_m^{N+2}}\braket{rs}{\chi_m^{N+2}}}{\omega  - \Omega_m^{N+2}}
		- \sum_{m} \frac{\braket{pq}{\chi_m^{N-2}}\braket{rs}{\chi_m^{N-2}}}{\omega  - \Omega_m^{N-2}}
\end{split}
\end{equation}
with
\begin{align}
	\braket{pq}{\chi_m^{N+2}} & = \sum_{c < d} \mel{pq}{}{cd} X_{cd}^{N+2,m} + \sum_{k < l}  \mel{pq}{}{kl} Y_{kl}^{N+2,m}
	\\
	\braket{pq}{\chi_m^{N-2}} & = \sum_{c < d} \mel{pq}{}{cd} X_{cd}^{N-2,m} + \sum_{k < l}  \mel{pq}{}{kl} X_{kl}^{N-2,m}
\end{align}
Although it looks more complicated, the same spin structure of the $T$-matrix has the than the dynamically-screened Coulomb potential in the $GW$ approximation.
For example, we have
\begin{equation}
\begin{split}
	T_{p_\sig q_\sigp r_\tau s_\taup}^{N+2}(\omega) 
		= \sum_{m} \frac{\braket{p_\sig q_\sigp}{\chi_m^{N+2}}\braket{q_\tau s_\taup}{\chi_m^{N+2}}}{\omega  - \Omega_m^{N+2}}
\end{split}
\end{equation}
with a similar structure for $T_{pqrs}^{N-2}(\omega)$, where
\begin{align}
	\begin{split}
		\braket{p_\sig q_\sig}{\chi_{m_{\sig\sig}}^{N+2}} 
		& = \sum_{c < d} \mel{p_\sig i_\sig}{}{c_\sig d_\sig} X_{c_\sig d_\sig}^{N+2,m_{\sig\sig}} 
		\\
		& + \sum_{k < l} \mel{p_\sig i_\sig}{}{k_\sig l_\sig} Y_{k_\sig l_\sig}^{N+2,m_{\sig\sig}}
	\end{split}
	\\
	\begin{split}
		\braket{p_\sig q_\bsig}{\chi_{m_{\sig\bsig}}^{N+2}} 
		& = \sum_{c < d} \braket{p_\sig q_\bsig}{c_\sig d_\bsig} X_{c_\sig d_\bsig}^{N+2,m_{\sig\bsig}} 
		\\
		& + \sum_{k < l} \braket{p_\sig q_\bsig}{k_\sig l_\bsig} Y_{k_\sig l_\bsig}^{N+2,m_{\sig\bsig}}
	\end{split}
	\\
	\begin{split}
		\braket{p_\sig q_\sig}{\chi_{m_{\sig\sig}}^{N-2}} 
		& = \sum_{c < d} \mel{p_\sig q_\sig}{}{c_\sig d_\sig} X_{c_\sig d_\sig}^{N-2,m_{\sig\sig}} 
		\\
		& + \sum_{k < l} \mel{p_\sig q_\sig}{}{k_\sig l_\sig} Y_{k_\sig l_\sig}^{N-2,m_{\sig\sig}}
	\end{split}
	\\
	\begin{split}
		\braket{p_\sig q_\bsig}{\chi_{m_{\sig\bsig}}^{N-2}} 
		& = \sum_{c < d} \braket{p_\sig q_\bsig}{c_\sig d_\bsig} X_{c_\sig d_\bsig}^{N-2,m_{\sig\bsig}} 
		\\
		& + \sum_{k < l} \braket{p_\sig q_\bsig}{k_\sig l_\bsig} Y_{k_\sig l_\bsig}^{N-2,m_{\sig\bsig}}
	\end{split}
\end{align}
are the only non-zero elements.
%========================================================
\subsection{Spin structure of the $T$-matrix self-energy}
%========================================================
In the spin-orbital basis, the correlation part of the $T$-matrix self-energy reads \cite{Zhang_2017}
\begin{equation}
	\Sigma^{T}_{pq}(\omega)
	= \sum_{im} \frac{\braket{pi}{\chi_m^{N+2}}\braket{qi}{\chi_m^{N+2}}}{\omega + \e{i}{} - \Omega_m^{N+2}}
	+ \sum_{am} \frac{\braket{pa}{\chi_m^{N-2}}\braket{qa}{\chi_m^{N-2}}}{\omega + \e{i}{} - \Omega_m^{N-2}}
\end{equation}

The  $T$-matrix self-energy $\Sigma^T_{\co,pq}$ can be spin-resolved
\begin{equation}
\begin{split}
	\Sigma^{T}_{p_\sig q_\sig}(\omega) 
	& = \sum_{i m} \frac{\braket{p_\sig i_\sig}{\chi_{m_{\sig\sig}}^{N+2}} \braket{q_\sig i_\sig}{\chi_{m_{\sig\sig}}^{N+2}}}{\omega + \e{i_\sig}{} - \Omega_{m_{\sig\sig}}^{N+2}}
	\\
	& + \sum_{i m} \frac{\braket{p_\sig i_\bsig}{\chi_{m_{\sig\bsig}}^{N+2}} \braket{q_\sig i_\bsig}{\chi_{m_{\sig\bsig}}^{N+2}}}{\omega + \e{i_\sig}{} - \Omega_{m_{\sig\bsig}}^{N+2}}
	\\
	& + \sum_{a m} \frac{\braket{p_\sig a_\sig}{\chi_{m_{\sig\sig}}^{N-2}} \braket{q_\sig a_\sig}{\chi_{m_{\sig\sig}}^{N-2}}}{\omega + \e{a_\sig}{} - \Omega_{m_{\sig\sig}}^{N-2}}
	\\
	& + \sum_{a m} \frac{\braket{p_\sig a_\bsig}{\chi_{m_{\sig\bsig}}^{N-2}} \braket{q_\sig a_\bsig}{\chi_{m_{\sig\bsig}}^{N-2}}}{\omega + \e{a_\sig}{} - \Omega_{m_{\sig\bsig}}^{N-2}}
\end{split}
\end{equation}
and is spin-diagonal like the $GW$ self-energy.



%================================
%\subsection{The dynamical screening}
%================================

%Based on the spin-up and spin-down components of $G$ defined in Eq.~\eqref{eq:G}, one can easily compute the non-interacting polarizability (which is a sum over spins)
%\begin{equation}
%\label{eq:chi0}
%	\chi_0(\br_1,\br_2;\omega) = - \frac{i}{2\pi} \sum_\sig \int G^{\sig}(\br_1,\br_2;\omega+\omega') G^{\sig}(\br_1,\br_2;\omega') d\omega'
%\end{equation}
%and subsequently the dielectric function
%\begin{equation}
%\label{eq:eps}
%	\epsilon(\br_1,\br_2;\omega) = \delta(\br_1 - \br_2) - \int \frac{\chi_0(\br_1,\br_3;\omega) }{\abs{\br_2 - \br_3}} d\br_3
%\end{equation}
%where $\delta(\br)$ is the Dirac delta function.
%Based on this latter ingredient, one can access the dynamically-screened Coulomb potential
%\begin{equation}
%\label{eq:W}
%	W(\br_1,\br_2;\omega) =  \int \frac{\epsilon^{-1}(\br_1,\br_3;\omega) }{\abs{\br_2 - \br_3}} d\br_3
%\end{equation}
%which is naturally spin independent as the bare Coulomb interaction $\abs{\br_1 - \br_2}^{-1}$ does not depend on spin coordinates.
%
%Within the $GW$ formalism, \cite{Hedin_1965,Onida_2002,Golze_2019} the dynamical screening is computed at the random-phase approximation (RPA) level by considering only the manifold of the spin-conserved neutral excitations.
%In the orbital basis, the spectral representation of $W$ is
%\begin{multline}
%\label{eq:W_spectral}
%	W_{p_\sig q_\sig,r_\sigp s_\sigp}(\omega) = \ERI{p_\sig q_\sig}{r_\sigp s_\sigp} 
%	+ \sum_m \ERI{p_\sig q_\sig}{m}\ERI{r_\sigp s_\sigp}{m} 
%	\\
%	\times \qty[ \frac{1}{\omega - \Om{m}{\spc,\RPA} + i \eta} - \frac{1}{\omega + \Om{m}{\spc,\RPA} - i \eta} ]
%\end{multline}
%where the bare two-electron integrals are \cite{Gill_1994}
%\begin{equation}
%		\ERI{p_\sig q_\tau}{r_\sigp s_\taup} = \int  \frac{\MO{p_\sig}(\br_1) \MO{q_\tau}(\br_1) \MO{r_\sigp}(\br_2) \MO{s_\taup}(\br_2)}{\abs{\br_1 - \br_2}} d\br_1 d\br_2
%\end{equation}
%and the screened two-electron integrals (or spectral weights) are explicitly given by
%\begin{equation}
%\label{eq:sERI}
%		\ERI{p_\sig q_\sig}{m} = \sum_{ia\sigp} \ERI{p_\sig q_\sig}{r_\sigp s_\sigp} (\bX{m}{\spc,\RPA}+\bY{m}{\spc,\RPA})_{i_\sigp a_\sigp}
%\end{equation}
%In Eqs.~\eqref{eq:W_spectral} and \eqref{eq:sERI}, the spin-conserved RPA neutral excitations $\Om{m}{\spc,\RPA}$ and their corresponding eigenvectors, $\bX{m}{\spc,\RPA}$ and $\bY{m}{\spc,\RPA}$, are obtained by solving a linear response system of the form
%\begin{equation}
%\label{eq:LR-RPA}
%	\begin{pmatrix}
%		\bA{}{}	&	\bB{}{}	\\
%		-\bB{}{}	&	-\bA{}{}	\\
%	\end{pmatrix}
%	\cdot
%	\begin{pmatrix}
%		\bX{m}{}	\\
%		\bY{m}{}	\\
%	\end{pmatrix}
%	=
%	\Om{m}{}
%	\begin{pmatrix}
%		\bX{m}{}	\\
%		\bY{m}{}	\\
%	\end{pmatrix}
%\end{equation}
%where the expressions of the matrix elements of $\bA{}{}$ and $\bB{}{}$ are specific of the method and of the spin manifold. 
%The spin structure of these matrices, though, is general 
%\begin{subequations}
%\begin{align}
%\label{eq:LR-RPA-AB-sc}
%	\bA{}{\spc} & = \begin{pmatrix}
%		\bA{}{\upup,\upup}	&	\bA{}{\upup,\dwdw}	\\
%		\bA{}{\dwdw,\upup}	&	\bA{}{\dwdw,\dwdw}	\\
%	\end{pmatrix}
%	&
%	\bB{}{\spc} & = \begin{pmatrix}
%		\bB{}{\upup,\upup}	&	\bB{}{\upup,\dwdw}	\\
%		\bB{}{\dwdw,\upup}	&	\bB{}{\dwdw,\dwdw}	\\
%	\end{pmatrix}
%\\
%\label{eq:LR-RPA-AB-sf}
%	\bA{}{\spf} & = \begin{pmatrix}
%		\bA{}{\updw,\updw}	&	\bO	\\
%		\bO					&	\bA{}{\dwup,\dwup}	\\
%	\end{pmatrix}
%	&
%	\bB{}{\spf} & = \begin{pmatrix}
%		\bO					&	\bB{}{\updw,\dwup}	\\
%		\bB{}{\dwup,\updw}	&	\bO	\\
%	\end{pmatrix}
%\end{align}
%\end{subequations}
%In the absence of instabilities, the linear eigenvalue problem \eqref{eq:LR-RPA} has particle-hole symmetry which means that the eigenvalues are obtained by pairs $\pm \Om{m}{}$.
%In such a case, $(\bA{}{}-\bB{}{})^{1/2}$ is positive definite, and Eq.~\eqref{eq:LR-RPA} can be recast as a Hermitian problem of half its original dimension 
%\begin{equation}
%\label{eq:small-LR}
%	(\bA{}{} - \bB{}{})^{1/2} \cdot (\bA{}{} + \bB{}{}) \cdot (\bA{}{} - \bB{}{})^{1/2} \cdot \bZ{}{} = \bOm{2} \cdot \bZ{}{}
%\end{equation}
%where the excitation amplitudes are
%\begin{equation}
%	\bX{}{} + \bY{}{} = \bOm{-1/2} \cdot (\bA{}{} - \bB{}{})^{1/2} \cdot \bZ{}{}
%\end{equation}
%Within the Tamm-Dancoff approximation (TDA), the coupling terms between the resonant and anti-resonant parts, $\bA{}{}$ and $-\bA{}{}$, are neglected, which consists in setting $\bB{}{} = \bO$.
%In such a case, Eq.~\eqref{eq:LR-RPA} reduces to a straightforward Hermitian problem of the form:
%\begin{equation}
%	\bA{}{} \cdot \bX{m}{} = \Om{m}{} \bX{m}{}
%\end{equation}
%Note that, for spin-flip excitations, it is quite common to enforce the TDA especially when one considers a triplet reference as the first ``excited-state'' is usually the ground state of the closed-shell system (hence, corresponding to a negative excitation energy).
%
%At the RPA level, the matrix elements of $\bA{}{}$ and $\bB{}{}$ are
%\begin{subequations}
%\begin{align}
%	\label{eq:LR_RPA-A}
%	\A{i_\sig a_\tau,j_\sigp b_\taup}{\RPA} & = \delta_{ij} \delta_{ab} \delta_{\sig \sigp}  \delta_{\tau \taup} (\e{a_\tau} - \e{i_\sig}) + \ERI{i_\sig a_\tau}{b_\sigp j_\taup}
%	\\ 
%	\label{eq:LR_RPA-B}
%	\B{i_\sig a_\tau,j_\sigp b_\taup}{\RPA} & = \ERI{i_\sig a_\tau}{j_\sigp b_\taup}
%\end{align}
%\end{subequations}
%from which we obtain the following expressions
%\begin{subequations}
%\begin{align}
%	\label{eq:LR_RPA-Asc}
%	\A{i_\sig a_\sig,j_\sigp b_\sigp}{\spc,\RPA} & = \delta_{ij} \delta_{ab} \delta_{\sig \sigp} (\e{a_\sig} - \e{i_\sig}) + \ERI{i_\sig a_\sig}{b_\sigp j_\sigp}
%	\\ 
%	\label{eq:LR_RPA-Bsc}
%	\B{i_\sig a_\sig,j_\sigp b_\sigp}{\spc,\RPA} & = \ERI{i_\sig a_\sig}{j_\sigp b_\sigp}
%\end{align}
%\end{subequations}
%for the spin-conserved excitations and 
%\begin{subequations}
%\begin{align}
%	\label{eq:LR_RPA-Asf}
%	\A{i_\sig a_\bsig,j_\sig b_\bsig}{\spf,\RPA} & = \delta_{ij} \delta_{ab} (\e{a_\bsig} - \e{i_\sig})
%	\\ 
%	\label{eq:LR_RPA-Bsf}
%	\B{i_\sig a_\bsig,j_\bsig b_\sig}{\spf,\RPA} & = 0
%\end{align}
%\end{subequations}
%for the spin-flip excitations.

%================================
%\subsection{The $GW$ self-energy}
%================================
%Within the acclaimed $GW$ approximation, \cite{Hedin_1965,Golze_2019} the exchange-correlation (xc) part of the self-energy
%\begin{equation}
%\label{eq:Sig}
%\begin{split}
%	\Sig{}{\xc,\sig}(\br_1,\br_2;\omega) 
%	& = \Sig{}{\x,\sig}(\br_1,\br_2) + \Sig{}{\co,\sig}(\br_1,\br_2;\omega) 
%	\\
%	& = \frac{i}{2\pi} \int G^{\sig}(\br_1,\br_2;\omega+\omega') W(\br_1,\br_2;\omega') e^{i \eta \omega'} d\omega'
%\end{split}
%\end{equation}
%is, like the one-body Green's function, spin-diagonal, and its spectral representation reads
%\begin{subequations}
%\begin{gather}
%	\Sig{p_\sig q_\sig}{\x}
%	= - \sum_{i} \ERI{p_\sig i_\sig}{i_\sig q_\sig}	 
%	\\
%\begin{split}
%	\Sig{p_\sig q_\sig}{\co}(\omega) 
%	& = \sum_{im} \frac{\ERI{p_\sig i_\sig}{m} \ERI{q_\sig i_\sig}{m}}{\omega - \e{i_\sig} + \Om{m}{\spc,\RPA} - i \eta}
%	\\
%	& + \sum_{am} \frac{\ERI{p_\sig a_\sig}{m} \ERI{q_\sig a_\sig}{m}}{\omega - \e{a_\sig} - \Om{m}{\spc,\RPA} + i \eta}			 
%\end{split}
%\end{gather}
%\end{subequations}
%where the self-energy has been split in its exchange (x) and correlation (c) contributions.
%The Dyson equation linking the Green's function and the self-energy holds separately for each spin component 
%\begin{equation}
%\label{eq:Dyson_G}
%\begin{split}
%	\qty[ G^{\sig}(\br_1,\br_2;\omega) ]^{-1} 
%	& = \qty[ G_{\KS}^{\sig}(\br_1,\br_2;\omega) ]^{-1} 
%	\\
%	& + \Sig{}{\xc,\sig}(\br_1,\br_2;\omega) - v^{\xc}(\br_1) \delta(\br_1 - \br_2)
%\end{split}
%\end{equation}
%where $v^{\xc}(\br)$ is the KS (local) exchange-correlation potential.
%The target quantities here are the quasiparticle energies $\eGW{p_\sig}$, \ie, the poles of $G$ [see Eq.~\eqref{eq:G}], which correspond to well-defined addition/removal energies (unlike the KS orbital energies).
%Because the exchange-correlation part of the self-energy is, itself, constructed with the Green's function [see Eq.~\eqref{eq:Sig}], the present process is, by nature, self-consistent.
%The same comment applies to the dynamically-screened Coulomb potential $W$ entering the definition of $\Sig{}{\xc}$ [see Eq.~\eqref{eq:Sig}] which is also constructed from $G$ [see Eqs.~\eqref{eq:chi0}, \eqref{eq:eps}, and \eqref{eq:W}].
%
%%================================
%\section{Unrestricted Bethe-Salpeter equation formalism}
%\label{sec:UBSE}
%%================================
%
%%================================
%\subsection{Static approximation}
%\label{sec:BSE}
%%================================
%
%Within the so-called static approximation of BSE, the Dyson equation that links the generalized four-point susceptibility $L^{\sig\sigp}(\br_1,\br_2;\br_1',\br_2';\omega)$ and the BSE kernel $\Xi^{\sig\sigp}(\br_3,\br_5;\br_4,\br_6)$ is \cite{ReiningBook,Bruneval_2016a}
%\begin{multline}
%	L^{\sig\sigp}(\br_1,\br_2;\br_1',\br_2';\omega) 
%	= L_{0}^{\sig\sigp}(\br_1,\br_2;\br_1',\br_2';\omega)
%	\\
%	+ \int L_{0}^{\sig\sigp}(\br_1,\br_4;\br_1',\br_3;\omega) 
%	\Xi^{\sig\sigp}(\br_3,\br_5;\br_4,\br_6)
%	\\
%	\times L^{\sig\sigp}(\br_6,\br_2;\br_5,\br_2';\omega)
%	d\br_3 d\br_4 d\br_5 d\br_6
%\end{multline}
%where 
%\begin{multline}
%	L_{0}^{\sig\sigp}(\br_1,\br_2;\br_1',\br_2';\omega) 
%	\\
%	= \frac{1}{2\pi} \int G^{\sig}(\br_1,\br_2';\omega+\omega') G^{\sig}(\br_1',\br_2;\omega') d\omega'
%\end{multline}
%is the non-interacting analog of the two-particle correlation function $L$.
%
%Within the $GW$ approximation, the static BSE kernel is
%\begin{multline}
%	\label{eq:kernel}
%	i \Xi^{\sig\sigp}(\br_3,\br_5;\br_4,\br_6) 
%	= \frac{\delta(\br_3 - \br_4) \delta(\br_5 - \br_6) }{\abs{\br_3-\br_6}} 
%	\\
%	- \delta_{\sig\sigp} W(\br_3,\br_4;\omega = 0) \delta(\br_3 - \br_6) \delta(\br_4 - \br_6) 
%\end{multline}
%where, as usual, we have not considered the higher-order terms in $W$ by neglecting the derivative $\partial W/\partial G$. \cite{Hanke_1980,Strinati_1982,Strinati_1984,Strinati_1988}
%
%As readily seen in Eq.~\eqref{eq:kernel}, the static approximation consists in neglecting the frequency dependence of the dynamically-screened Coulomb potential.
%In this case, the spin-conserved and spin-flip BSE optical excitations are obtained by solving the usual Casida-like linear response (eigen)problem:
%\begin{equation}
%\label{eq:LR-BSE}
%	\begin{pmatrix}
%		\bA{}{\BSE}		&	\bB{}{\BSE}	\\
%		-\bB{}{\BSE}	&	-\bA{}{\BSE}	\\
%	\end{pmatrix}
%	\cdot
%	\begin{pmatrix}
%		\bX{m}{\BSE}	\\
%		\bY{m}{\BSE}	\\
%	\end{pmatrix}
%	=
%	\Om{m}{\BSE}
%	\begin{pmatrix}
%		\bX{m}{\BSE}	\\
%		\bY{m}{\BSE}	\\
%	\end{pmatrix}
%\end{equation}
%Defining the elements of the static screening as $W^{\stat}_{p_\sig q_\sig,r_\sigp s_\sigp} = W_{p_\sig q_\sig,r_\sigp s_\sigp}(\omega = 0)$, the general expressions of the BSE matrix elements are
%\begin{subequations}
%\begin{align}
%	\label{eq:LR_BSE-A}
%	\A{i_\sig a_\tau,j_\sigp b_\taup}{\BSE} & = \A{i_\sig a_\tau,j_\sigp b_\taup}{\RPA} - \delta_{\sig \sigp} W^{\stat}_{i_\sig j_\sigp,b_\taup a_\tau}
%	\\ 
%	\label{eq:LR_BSE-B}
%	\B{i_\sig a_\tau,j_\sigp b_\taup}{\BSE} & = \B{i_\sig a_\tau,j_\sigp b_\taup}{\RPA} - \delta_{\sig \sigp} W^{\stat}_{i_\sig b_\taup,j_\sigp a_\tau}
%\end{align}
%\end{subequations}
%from which we obtain the following expressions for the spin-conserved and spin-flip BSE excitations: 
%\begin{subequations}
%\begin{align}
%	\label{eq:LR_BSE-Asc}
%	\A{i_\sig a_\sig,j_\sigp b_\sigp}{\spc,\BSE} & = \A{i_\sig a_\sig,j_\sigp b_\sigp}{\spc,\RPA} - \delta_{\sig \sigp} W^{\stat}_{i_\sig j_\sigp,b_\sigp a_\sig}
%	\\ 
%	\label{eq:LR_BSE-Bsc}
%	\B{i_\sig a_\sig,j_\sigp b_\sigp}{\spc,\BSE} & = \B{i_\sig a_\sig,j_\sigp b_\sigp}{\spc,\RPA} - \delta_{\sig \sigp} W^{\stat}_{i_\sig b_\sigp,j_\sigp a_\sig}
%	\\
%	\label{eq:LR_BSE-Asf}
%	\A{i_\sig a_\bsig,j_\sig b_\bsig}{\spf,\BSE} & = \A{i_\sig a_\bsig,j_\sig b_\bsig}{\spf,\RPA} - W^{\stat}_{i_\sig j_\sig,b_\bsig a_\bsig}
%	\\ 
%	\label{eq:LR_BSE-Bsf}
%	\B{i_\sig a_\bsig,j_\bsig b_\sig}{\spf,\BSE} & = - W^{\stat}_{i_\sig b_\sig,j_\bsig a_\bsig}
%\end{align}
%\end{subequations}
%
%At this stage, it is of particular interest to discuss the form of the spin-flip matrix elements defined in Eqs.~\eqref{eq:LR_BSE-Asf} and \eqref{eq:LR_BSE-Bsf}.
%As readily seen from Eq.~\eqref{eq:LR_RPA-Asf}, at the RPA level, the spin-flip excitations are given by the difference of one-electron energies, hence missing out on key exchange and correlation effects.
%This is also the case at the TD-DFT level when one relies on (semi-)local functionals. 
%This explains why most of spin-flip TD-DFT calculations are performed with hybrid functionals containing a substantial amount of Hartree-Fock exchange as only the exact exchange integral of the form $\ERI{i_\sig j_\sig}{b_\bsig a_\bsig}$ survive spin-symmetry requirements.
%At the BSE level, these matrix elements are, of course, also present thanks to the contribution of $W^{\stat}_{i_\sig j_\sig,b_\bsig a_\bsig}$ as evidenced in Eq.~\eqref{eq:W_spectral} but it also includes correlation effects.

%%%%%%%%%%%%%%%%%%%%%%%%
\bibliography{Tmatrix}
%%%%%%%%%%%%%%%%%%%%%%%%

\end{document}
