\documentclass[10pt]{letter}
\usepackage{UPS_letterhead,xcolor,mhchem,mathpazo,ragged2e}
\newcommand{\alert}[1]{\textcolor{red}{#1}}
\definecolor{darkgreen}{HTML}{009900}


\begin{document}

\begin{letter}%
{To the Edi	tors of the Journal of Chemical Physics}

\opening{Dear Editors,}

\justifying
Please find enclosed our manuscript entitled \textit{``Static and Dynamic Bethe-Salpeter Equations in the $T$-Matrix Approximation''}, which we would like you to consider as a Regular Article in the \textit{Journal of Chemical Physics}.
This contribution has never been submitted in total nor in parts to any other journal, and has been seen and approved by all authors.

The many-body Green's function Bethe-Salpeter equation (BSE) formalism is steadily asserting itself as a new efficient and accurate tool in the ensemble of computational methods available to chemists in order to predict optical excitations in molecular systems.
The BSE formalism usually relies on the well-known $GW$ approximation (which corresponds to an elegant resummation of all direct ring diagrams from the particle-hole channel and is particularly justified in the high-density or weakly-correlated regime) to compute the fundamental gap and excitonic effect. 
Here, we study an interesting alternative better suited in the low-density or strongly-correlated regime known as the $T$-matrix (or Bethe-Goldstone) approximation where one sums to infinity the ladder diagrams from the particle-particle channel.
 
We have derived and implemented, for the first time, the static and dynamic Bethe-Salpeter equations when one considers $T$-matrix quasiparticle energies as well as a $T$-matrix-based kernel. 
The performance of the static scheme and its perturbative dynamical correction have been assessed by computing the neutral excited states of several molecular systems and comparison with standard wave function methods as well as the $GW$-based formalism are presented.
Our results suggest that the present $T$-matrix scheme performs best in few-electron systems where the electron density remains low.
For larger systems, it seems to be, overall, less accurate than BSE@$GW$ but still outperforms conventional methods such CIS and TDHF for singlet states.

Because of the novelty of this work and its potential impact in quantum chemistry and condensed matter physics, we expect it to be of interest to a wide audience within the chemistry and physics communities.
We suggest Xavier Blase, Timothy Berkelbach, Weitao Yang, Wim Klopper, Fabien Bruneval, Patrick Rinke, and Lucia Reining as potential referees.
We look forward to hearing from you soon.

\closing{Sincerely, the authors.}

\end{letter}
\end{document}






